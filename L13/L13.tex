\documentclass[../Main.tex]{subfiles}
\setcounter{chapter}{12}

\begin{document}
\chapter{L13: Polynomial Rings over UFDs}

\begin{lem}[title = Gauss's Lemma,label=gauss]
	Let $R$ be a UFD and $F$ its field of fractions. Let $p(X)\in R[X]$, then if $p(X)$ is reducible in $F[X]$ then $p(X)$ is reducible in $R[X]$.\\
	Explicitly, if $p(X)=A(X)\rdot B(X)$ and $A\rdot B\in F[X]$, then there exist $r,s\in F$ such that 
	\[r\rdot A(X) = a(X)\in R[X],\quad s\rdot B(X) = b(X) \in R[X]\]
	and $p(X) = a(X)\rdot b(X)$.
\end{lem}
\Obs that $F[X]^\times = F$, i.e the constant polynomials. Then since $p(X)$ is reducible, $A(X)$ and $B(X)$ are non-units, and hence
\[A(X),B(X) \notin F[X]^\times \implies \deg A, \deg B \ge 1\]
\begin{example}
	Consider the polynomial
	\[15X^2+13X+2 = \underbrace{\left(\frac{5}{2}X + \frac{5}{3}\right)}_{A(X)}\rdot  \underbrace{\left(6X + \frac{6}{5}\right)}_{B(X)}\]
	Then see that by looking to clear the denominators of $A(X)$ and $B(X)$ we get,
	\begin{align*}
	2\rdot 3\rdot 5(15X^2+13X+2) &= \left[2\rdot 3 \rdot \left(\frac{5}{2}X + \frac{5}{3}\right) \right] \rdot  \left[5\rdot \left(6X + \frac{6}{5}\right)\right] \\
	&= (15X+10) \rdot{}  (30X+6)
	\end{align*}
	Now we have factored the a multiple of our polynomial, so we get back to the original polynomial by dividing $2\rdot 3\rdot 5$ in such a way that we redistribute where they end up
	\begin{align*}
	15X^2+13X+2&=\left[\underbrace{\frac{2\rdot 3}{5}}_{r}\underbrace{\left(\frac{5}{2}X + \frac{5}{3}\right)}_{A(X)}\right]\rdot \left[\underbrace{\frac{5}{2\rdot 3}}_{s} \underbrace{\left(6X + \frac{6}{5}\right)}_{B(X)}\right]\\
	&= (\underbrace{3X+2}_{a(X)})\rdot{}  (\underbrace{5X+1}_{b(X)})
	\end{align*}
\end{example}	
\begin{proof}~\\
	Write out the polynomials $A(X),B(X)$ where $\deg A(X) =n$ is not necessarily equal to $\deg B(X)=m$,
	\begin{align*}
	A(X)=\frac{a_0}{\alpha_0}+\frac{a_1}{\alpha_1}X_1+\dots +\frac{a_n}{\alpha_n}X^n\\
	B(X)= \frac{b_0}{\beta_0}+\frac{b_1}{\beta_1}X_1+\dots +\frac{b_m}{\beta_m}X^m
	\end{align*}
	We want to clear out the denominators, so let
	\[\begin{rcases*}
	\alpha = \alpha_0\alpha_1\dots\alpha_n\\
	\beta = \beta_0\beta_1\dots\beta_m
	\end{rcases*}d=\alpha\rdot \beta\]
	(1) Since $R$ is an integral domain and none of the $\alpha_i$'s and $\beta_i$'s can be $0$ (as they are in denominators of fractions), so $\alpha,\beta,d\ne 0$\\
	(2) Now after clearing out the denominators, denote the new polynomials 
	\[\begin{array}{l}
		\alpha\rdot A(X)=a'(X)\\[2pt]
		\beta\rdot B(X)=b'(X)
	\end{array}\in R[X]\]
	For example
	\begin{align*}
	\underbrace{(2\rdot 3)}_{\alpha}\rdot \underbrace{\left(\frac{5}{2}X+\frac{5}{3}\right)}_{A(X)}&=\underbrace{15X+10}_{a'(X)}\\ 
	\underbrace{5}_{\beta}\rdot \underbrace{\left(6X+\frac{6}{5}\right)}_{B(X)} &= \underbrace{30X+6}_{b'(X)}
	\end{align*}
	Therefore $d\rdot p(X)=a'(X)\rdot b'(X)$.\\
	Write
	$d=q_1\rdot q_2\rdot \dots\rdot q_k,$ where $q_i$ is irreducible $\forall i\in \{1,\dots,k\} $.
	Then $(q_i)\subset R$ is prime, hence
	\begin{align*}
	R[X]/q_iR[X] \cong (R/(q_i))[X] \text{ is an integral domain}
	\end{align*}
	Furthermore,
	\[\divs{q_i}{d}\implies \obar{d\rdot p(X)}=\obar{0} \in (R/(q_i))[X] \implies \obar{a'(X)}\rdot \obar{ b'(X)} =\obar{0}\]
	Since $a'(X)$ or $b'(X)$ are equal to the $0$ coset, then it is equivalent to say $a'(X)$ or $b'(X)$ are in $q_iR[X]$ (the ideal being modded out). In other words, whichever of the two is equal to $\bar{0}$ will have $q_i$ as a factor of the numerators of their coefficients. Therefore
	\[\frac{1}{q_i}\rdot a'(X) \text{ or } \frac{1}{q_i}\rdot b'(X) \in R[X]\] 
	Now assuming w.l.o.g. it is $a'(X)$ which has $q_i$ then
	 \[\frac{d}{q_i}\rdot p(X)=\underbrace{\left[\frac{1}{q_i}\rdot a'(X)\right]}_{\in R[X]}\rdot \underbrace{b'(X)}_{\in R[X]} \]
	If we continue doing this process for all the irreducibles that appear in the factorization of $d$, then eventually we will clear all of $d$ on the left, and at each stage we are ending up with polynomials in $R[X]$. So, in the end we get
	\[p(X)=\underbrace{a(X)}_{\in R[X]}\rdot \underbrace{b(X)}_{\in R[X]}\qedhere\]
\end{proof}
Going back to the previous example, what we were doing is
\begin{align*}
30\rdot p(X)&=(15X+10)\rdot (30X+6)\\
15\rdot p(X)&=(15X+10)\rdot (15X+3)\\
3\rdot p(X)&=(3X+2)\rdot (15X+3)\\
p(X) &= (3X+2)\rdot (5X+1)\qedhere
\end{align*}
To rephrase Gauss's Lemma in the form of its contrapositive: \\ \hyperref[lem:gauss]{If $p(X)$ is irreducible in $R[X]$, then it is \textbf{still} irreducible in $F[X]$}.  The point being that if $R$ is a UFD and $F$ is its field of fractions, knowing that $p(X)$ is irreducible in $R[X]$ and adding structure to reach $F[X]$ isn't enough structure to make $p(X)$ reducible.

\textbf{Q:} Are there any irreducibles in $F[X]$ that \textbf{are not} irreducible in $R[X]$?\\
\textbf{Recall} that if $F,K$ are fields with $F\subset K$ then \[p(X) \text{ irreducible } \in F[X] \Longleftrightarrow p(X) \text{ irreducible }\in K[X]\]
So in a more general setting with fields, it is not the case. So let us to continue consider our case where $R$ is a UFD, to which the answer is yes.
\begin{example}
	$7X$ is reducible in $\Z[X]$ as $7$ and $X$ are non-units. But $7\in \Q^\times$, so $7,X$ do not constitute a reduction of $7X$ in $\Q[X]$. Now it could be the case that $7X$ is reducible in another way not involving $7$ and $X$, but we can prove in fact that there \textbf{isn't} a way of writing $7X$ as the product of two irreducibles in $\Q[X]$.
	\begin{proof}~\\ $7X$ is associate to $X$ (only differ by a unit) and notably $\Q[X]/(X)\cong \Q$ and since $\Q$ is a field, then
	\[ (X) \text{ is maximal}\implies (X) \text{ is prime}\implies X\text{ is irreducible}\implies 7X\text{ is irreducible}\]
	where the last implication is since $7$ is associate to $X$ then since $7$ is a unit and $X$ is irreducible (hence not a unit), $7X$ is irreducible.
	\end{proof}
\end{example}
In fact, we see that by shifting to the field of fractions, one of the elements in $7X$ became a unit, namely $7$. As a corollary to \hyperref[lem:gauss]{Gauss's Lemma}, we will see how situations like this are the only things that turn from irreducibles to units as one goes to the field of fractions. 
\begin{crl}[label=13.2]
	Let $R$ be a UFD and $F$ its field of fractions. If
	\[p(X)=a_0+a_1X+\dots+a_nX^n \in R[X]\]
	and $\gcd(a_0,a_1,\dots,a_n)=1$. Then
	\[p(X) \text{ irreducible }\in R[X] \Longleftrightarrow p(X) \text{ irreducible }\in F[X]\]
	\Note $\gcd(a_0,a_1,\dots,a_n)=1$ means we cannot factor out a non-unit from the coefficients, i.e. we cannot write
	\[p(X)=d\rdot p'(X),\quad d\in R\setminus R^\times, \quad \deg p =\deg p'\] 
\end{crl}
\begin{proof}~\\
	This will be proved by contrapositive. In the first direction, it is to show that if $p(X)$ is reducible in $F[X]$ then it is reducible in $R[X]$
	Suppose $p(X)\in R[X]$ is reducible in $R[X]$ and $\gcd(a_0,a_1,\dots,a_n)=1$. That is, suppose
	\[p(X)=a(X)\rdot b(X),\quad a(X), b(X) \notin R[X]^\times \]
	Then since
	$\gcd(a_0,a_1,\dots,a_n)=1$ the note in the statement of the corollary essentially says $a(X),b(X)$ are non-constant polynomials because you can not factor out of $p(X)$ a constant non-unit. So in fact that means $\deg a, \deg b\ge 1$.\\
	However, we know $F[X]^\times$ is exactly $F^\times$, the non-zero constant polynomials. Hence $a(X),b(X)\in F[X]$ are not units in $F[X]$ and so $p(X)$ is reducible in $F[X]$.\\
	The other direction is \hyperref[lem:gauss]{Gauss's Lemma}.
\end{proof}
\begin{thm}[title = \texorpdfstring{$R$ UFD $\Longleftrightarrow R[X]$ UFD}{R UFD iff R[X] UFD}]
	$R$ is a UFD if and only if $R[X]$ is a UFD.
\end{thm}
\begin{proof}~\\
	$\Leftarrow$\\
	If $R[X]$ is a UFD, then since $R\subset R[X]$ is a subring then $R$ is also a UFD.\\
	$\Rightarrow$\\
	Suppose that $R$ is a UFD and $F$ is its field of fractions. We can write
	\[p(X) =a_0+a_1X+\dots+a_nX^n\in R[X]\]
	The goal is to uniquely factor $p(X)$ in $R[X]$. Let
	\[d=\gcd(a_0,a_1,\dots,a_n)\in R\]
	If $d\notin R^\times$, then it has unique factorization into irreducibles in $R$ (since $R$ is a UFD) and necessarily $p(X) = d\rdot p'(X)$ where the $\gcd$ of the coefficients in $p'(X)$ is 1. \\
	Now assume $\gcd(a_0,a_1,\dots,a_n)=1$; in particular, if $p(X)\notin R[X]^\times$ then $\deg p\ge 1$.\\
	Consider $p(X)\in F[X]$ and note the $F[X]$ is a UFD (actually a Euclidean domain). This implies we can write
	\[p(X)=A_1(X)\rdot A_2(X)\rdot \dots \rdot A_k(X)\]
	where $A_i(X)\in F[X]$ are irreducible. By \hyperref[lem:gauss]{Gauss's Lemma} we can clear out the denominators and write
	\[p(X)=a_1(X)\rdot a_2(X)\rdot \dots \rdot a_k(X)\]
	where $a_i(X)\in R[X]$. Then
	\[\gcd(a_0,\dots,a_n)=1 \implies \gcd(\text{coeffs of }a_i(X))=1\quad \forall i\]
	By \hyperref[co:13.2]{Corollary 13.2}, since $a_i(X)\in R[X]$ is associate to $A_i(X)$ in $F[X]$, hence $a_i(X)$ is irreducible in $R[X]$. So we've shown there exists a factorization of $p(X)$ as a product of irreducibles in $R[X]$.\\
	The uniqueness follows directly from uniqueness in $F[X]$.
\end{proof}
\end{document}