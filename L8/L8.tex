\documentclass[../Main.tex]{subfiles}
\setcounter{chapter}{7}

\begin{document}
\chapter{L8: Maximal Ideals and Ring of Fractions}

\textit{Recall}: $(X) \subset \Z[X]$ is prime, but $(X)\subsetneq (2,X)$, so it not maximal.\\
$(X)\in \R[X]$ is maximal because $\R[X]/(X) \cong \R$ is a field

\begin{example}
	Let $a\in \R$. We defined the evaluation homomorphism before:
	\begin{align*}
	\ev_a\colon \R[X] &\to \R \\
	p(X) &\mapsto p(a)
	\end{align*}
	\textit{Observe} that $\ev_a$ is in fact surjective. Then
	\[\R[X]/\Ker(\ev_a) \cong \R \implies \Ker(\ev_a) \text{ is a maximal ideal}\]
\end{example}
	Denote $M_a\coloneqq \Ker(\ev_a)$
	\begin{claim}
		$M_a = (X-a)$\quad (e.g $M_0 =(X)$)
	\end{claim}
	\begin{proof}~\\
		If $p(X) \in (X-a)$ then we may write $p(X)=q(X)\rdot (X-a),\, q(X)\in \R[X]$, then
		\[\ev_a(p(X)) = p(a) = q(a)\rdot (a-a)=0 \implies p(X)\in M_a \implies (X-a) \subset M_a\] 
		Conversely, suppose $p(X) \in M_a = \Ker(\ev_a)$. Let $p(X) = a_0 + a_1X+a_2X^2+\dots+a_nX^n$, then you can check with polynomial division that $X-a$ divides $p(X)$ with remainder exactly $p(a)$ which is $0$, hence $X-a$ is a factor of $p(X)$ [obviously, if $p(X)$ is a polynomial with a root at $X=a$, then $X-a$ is a factor], and we can write
		\[\frac{p(X)}{X-a} = q(X)\]
		therefore,
		\[p(X)=q(X)\rdot (X-a)\implies p(X) \in (X-a) \implies M_a \subset (X-a) \]
		and hence $M_a =(X-a)$.
	\end{proof}
\textbf{\textit{Q:}} Is every maximal ideal of $\R[X]$ of the form $M_a$?\\
For example, in $\Z$, the $\{\text{maximal ideals}\} =\{\text{prime ideals}\}$ but we saw above that in $\Z[X]$ there exist prime ideals that are not maximal.\\
Two standard questions:\\
(1) What are the primes?\\
(2) What are the maximal ideals?

\begin{claim}
Consider $I=(X^2+1)$, then $I\subset \R[X]$ is a maximal ideal.
\end{claim}
\begin{proof}
	We have that
	\[\R[X] = \{a_0+a_1X+a_2X^2+a_3X^3+\dots+a_nX^n| a_k\in \R, \, k=0,1,2,\dots n\}\]
	What does $\obar{X^n}$ look like in $\R[X]/(X^2+1)$?
	We can deduce from the zero coset of the ideal:
		\[X^2+1 \in (X^2+1) \implies \obar{X^2+1} = \obar{0} \implies \obar{X^2} = \obar{-1} \in \R[X]/I \]
	Furthermore
		\begin{align*}
		X^3 = X\rdot X^2 &\implies \obar{X^3} = \obar{X} \rdot \obar{(-1)} \in \R[x]/I\\
		X^4= X^2\rdot x^2 &\implies \obar{X^4} = \obar{(-1)} \rdot  \obar{(-1)} \in \R[X]/I
	\end{align*}
	Therefore, since all powers of $X$ greater than $2$ can be deconstructed into products of $-1$ and $X$, we can collapse the cosets of the quotient to a convenient form:
	\[\R[X]/I = \{\obar{a_0+a_1X}\mid a_0,a_1\in \R\} \]
	with the rule $\obar{X}\rdot \obar{X}=\obar{-1}$.\\
	This should be familiar and there is a ring isomorphism
	\begin{align*}
	\R[X]/I &\to \C \\
	\obar{1} &\mapsto 1 \\
	\obar{X} &\mapsto i
	\end{align*}
	and since the quotient ring is isomorphic to the field $\C$, $I$ is maximal.
\end{proof}
\begin{claim}
	$(X^2+1)$ is \textbf{not} maximal in $\C[X]$
\end{claim}
\begin{proof}
	We know that $X+i,x-i \in \C[X]$ and
	\[(X+i)(X-i)=X^2+1\in (X^2+1)\]
	But $X+i,X-i \notin (X^2+1)$ therefore $(X^2+1)$ is not prime in $\C[X]$ and consequently is not maximal.
\end{proof}
\textit{Observe} if $a\in R\subset S$ Then
\begin{align*}
(a&)_R = \{r\rdot a | r\in R\}\\
\cap&\\
(a&)_S=\{s\rdot a|s\in S\}
\end{align*}
can have different properties as ideals, e.g
\begin{align*}
\underbrace{(X)\subset \Z[X]}_{\text{prime}} &\longrightarrow \,\,\underbrace{(x) \subset \R[X]}_{\text{maximal}}
\\
\underbrace{(X^2+1)\subset \R[X]}_{\text{maximal}} &\longrightarrow \underbrace{(X^2+1) \subset \C[X]}_{\text{not prime, not maximal}}
\end{align*}\newpage
\section*{The Ring of Fractions}
\textbf{\textit{Q:}} How do we build $\Q$ out of $\Z$?\\
We want to add in multiplicative inverses like $\frac{1}{2},\frac{1}{3},\frac{1}{4},\dots$ but we can't just add them in and get a ring.\\
Consider
\[\Z \times (\Z\setminus\{0\}) =\{(m,n)\mid m,n\in\Z,n\ne0\} \]
and think of the elements of this set as the fractions $\frac{m}{n}$.\\
There are some repeats if we care about multiplication and addition like 
\[\frac{1}{2}=\frac{2}{4}=\frac{3}{6}\]
We should define an equivalence relation
\[\frac{a}{b}\sim \frac{c}{d} \Longleftrightarrow ad=bc\]
e.g $\frac{4}{6}\sim \frac{6}{9}$ because $4\rdot 9=36=6\rdot 6$.
\begin{dfn}[title = Field of Rational Numbers]
	The \textbf{field of rational numbers} is 
	\[\Q\coloneqq \left\{\left.\frac{m}{n}\right|\, m,n\in\Q,\,n\ne0\right\}/\sim \]
	and this is a field with operations given by
	\begin{align*}
	\frac{a}{b}+\frac{c}{d} &= \frac{ad+bc}{bd}\\
	\frac{a}{b}\rdot \frac{c}{d} &= \frac{ac}{bd}
	\end{align*}
\end{dfn}
We can also see that there is an injective ring homomorphism
\begin{align*}
\Z &\to \Q \\
n &\mapsto \frac{n}{1}
\end{align*}
\begin{claim}
	If $F$ is a field and there is an injective ring homomorphism
	\[f\colon \Z \to F\]
	Then it factors through $\Q$, i.e there is a ring homomorphism
	\[\obar{f}\colon \Q \to F \text{ such that } f(n)=\obar{f}\left(\frac{n}{1}\right)\]
	\begin{center}
	\begin{tikzcd}[column sep=small]
		\Z \arrow[rr, "i"] \arrow[rd, "f"] &  & \Q \arrow[dl, dashed,"\obar{f}"]\\
		& F  & 
	\end{tikzcd}
	\end{center}
\end{claim}
This is basically saying that if you have an injective homomorphism from $Z$ to a field $F$, then under the homomorphism the integers will have inverses $f(2)\rdot \frac{1}{2} \in F$ and one should see that this is exactly the rationals $\Q$ existing inside $F$.\\
Suppose $R$ is any commutative ring with $1\ne 0$.\\
\textbf{\textit{Q:}} Can we do something similar with general rings $R$? i.e
\[R \times (R\setminus\{0\}) =\{(r,s)\mid r,s\in R,s\ne0\} \]
(again, we will write $(r,s)$ as $\frac{r}{s}$). 
We want to define $r^{-1}=\frac{1}{r},\, r\ne 0$.\\
However, if $r$ is a zero divisor, $r\rdot s=0$ then in this case we want to exclude
\[\frac{1}{r}\rdot \frac{1}{s}=\frac{1}{r\rdot s} = \frac{1}{0}\]
\begin{dfn}[title =Field of Fractions]
	Let $R$ be an integral domain with $1\ne 0$. Consider
	\[R\times (R\setminus\{0\}) = \{(r,s)\mid r,s\in R,s\ne 0\}\]
	Define an equivalence relation (\textbf{exercise to show it is}) by 
	\[\frac{a}{r}\sim \frac{b}{s} \Longleftrightarrow a\rdot s=b\rdot r\]
	There is no ambiguity in the equality of products since $R$ is integral there are no zero zero divisors, $s,r\ne 0$.\\
	The \textbf{field of fractions} of $R$ is
	\[Q(R) \coloneqq R\times (R\setminus\{0\})/{\sim} \,= \left\{\left.\left[\frac{a}{b}\right]\right|a,b\in R,\,b\ne 0\right\} \]
\end{dfn}
\begin{thm}
	$Q(R)$ is a field with operations
	\begin{align*}
	\frac{a}{r}+\frac{b}{s} = \frac{as+br}{rs},\qquad 
	\frac{a}{r}\rdot \frac{b}{s} = \frac{ab}{rs}
	\end{align*}
	The map 
	\begin{align*}
	i\colon R&\to Q(R)\\
	r&\mapsto \frac{r}{1}
	\end{align*}
	is an injective ring homomorphism (we say $R$ is a subring of its field of fractions).\\
	Moreover, if $F$ is any field such that $R\subset F$ is a subring (i.e there exists an injective ring homomorphism $f\colon R\to F$), then there is a ring homomorphism
	\[\obar{f}\colon Q(R) \to F \text{ such that } f(x)=\obar{f}\circ i(x)\]
	\begin{center}
		\begin{tikzcd}[column sep=small]
			R \arrow[rr, "i"] \arrow[rd, "f"] &  & Q(R) \arrow[dl, dashed,"\obar{f}"]\\
			& F  & 
		\end{tikzcd}
	\end{center}
\end{thm}
\begin{proof}
	Think about it\dots...
\end{proof}
\newpage
\begin{example}
	$Q(\Z)=\Q$
\end{example}
\begin{example}
	$R=\R[X]$ is an integral domain. The fractional field of $R$ is the field of rational functions
	\[Q(R)=\R(X)\coloneqq \left\{\left. \frac{p(X)}{q(X)}\right|p,q\in \R[X],\, q\ne 0 \right\} \]
\end{example}
\begin{example}
	If $R$ is any integral domain with field of fractions $Q(R)=F$.  Consider the integral domain $R[X]$. Then in particular $R\subset R[X]$, and  $R[X]\subset Q(R[X])$ which tells us that
	\begin{center}
		\begin{tikzcd}[column sep=small]
			R \arrow[rr, "\text{inclusion}"] \arrow[rd] &  & Q(R[X]) \\
			& F \arrow[ur] & 
		\end{tikzcd}
	\end{center}
	e.g $\Z \subset \Z[X]$, so in particular $\Q\subset Q(\Z[X])$.\\
	In fact, since in $Q(\Z[X])$ you've added inverses to the coefficients but you also inverses to the polynomials, so you will get the field of rational functions
	\[Q(\Z[X])=\R(X)\]
	Furthermore, this is generally true, as the field of fractions of $R[X]$ is going to be the rational functions with coefficients in the field of fractions of $R$, i.e
	\[Q(R[X])=F(X)\]
\end{example}
\end{document}