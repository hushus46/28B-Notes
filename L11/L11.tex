\documentclass[../Main.tex]{subfiles}
\setcounter{chapter}{10}
\begin{document}
\chapter{L11: Unique Factorization Domains}
\begin{dfn}[title = {Irreducible/Reducible, Prime, Associate Elements}]
	Let $R$ be an integral domain
	\begin{enumerate}
		\item Suppose $r\in R\setminus\{0\},\, r\notin R^\times$.\\
		We say $r$ is \textbf{irreducible} if whenever $r=a\rdot b$, either $a\in R^\times$ or $b\in R^\times$.\\
		We say $r$ is \textbf{reducible} if it is not irreducible.
		\item Suppose $r\in R\setminus\{0\},\, r\notin R^\times$\\
		We say $r$ is \textbf{prime} if $(r)$ is a prime ideal.\\
		In other words, if $\divs{r}{a\rdot b}$, then either $\divs{r}{a}$ or $\divs{r}{b}$.
		\item We say $a,b \in R$ are \textbf{associates} if there exists $u\in R^\times$ such that $a=u\rdot b$.\\ (If $a$ and $b$ generate the same principal ideal, then they are associates. Check HW3 P1)
	\end{enumerate}
\end{dfn}

\begin{prop}[title = Prime elements in integral domain are irreducible,label=11.2]
	Any prime element in an integral domain is irreducible.
\end{prop}
\begin{proof}
	Suppose $p=a\rdot b\in R$ and $(p)$ is a prime ideal.\\
	Then $p\in (p)$ implies $a\in (p)$ or $b\in (p)$. W.l.o.g let $a\in (p)$.\\
	So $\exists r\in R$ such that $a=p\rdot r$ and hence
	\[p=(p\rdot r)\rdot b=p\rdot (r\rdot b)\]
	Since $R$ is an integral domain, we can \hyperref[prop:cancel]{cancel} $p$, then, $1=r\rdot b$, so $b\in R^\times $.
\end{proof}
\begin{example}
	It turns out to be that the converse is not true, i.e irreducible but not prime elements.\\
	Consider the ring 
	\[\Z[\sqrt{-5}] \coloneq \{a+b\sqrt{-5}\mid a,b\in \Z\} \]
	There exists a norm on this (not exactly the same as the \hyperref[dfn:norm]{norm} for Euclidean domains), with properties
	\begin{itemize}
		\item $N(a+b\sqrt{-5}) \coloneqq a^2+5b^2$
		\item $N(x\rdot y)=N(x)\rdot N(y)$
		\item $N(x)=\pm 1$ if an only if $x\in \Z[\sqrt{-5}]^\times$
	\end{itemize}
\end{example}
\begin{claim}
	$2+\sqrt{-5}$ is irreducible
\end{claim}
\begin{proof}
	Suppose 
	\[2+\sqrt{-5}=(a+b\sqrt{-5})\rdot (c+d\sqrt{-5})\]
	Then
	\[N(2+\sqrt{-5}) = 4+5 =9 \implies \divs{N(a+b\sqrt{-5})}{9} \implies N(a+b\sqrt{-5}) = \pm 1 \text{ or } \pm 3\]
	\Obs that if $b\ne 0$, then \[N(a+b\sqrt{-5}) = a^2+5b^2\ge 5\]
	Therefore we have to assume $b=0$ if we want the norm to be $\pm 1, \pm 3$ and so
	\[b=0\implies N(a+b\sqrt{-5})=N(a)=a^2\]
	So the norm is a perfect square, and since the only candidates are $\pm 1$ and $\pm 3$, the only perfect square between them is $1$ and hence
	\[N(a+b\sqrt{-5})=1 \implies a+b\sqrt{-5}\in \Z[\sqrt{-5}]^\times\]
	which means that $2+\sqrt{-5}$ is irreducible.
\end{proof}
\begin{claim}
	$2+\sqrt{-5}$ is \textbf{not} prime.
\end{claim}
\begin{proof}
	We know
	\[3^2=9=(2+\sqrt{-5})\rdot (2-\sqrt{-5}) \in (2+\sqrt{-5})\]
	It remains to show that $3$ is not in the ideal $(2+\sqrt{-5})$, i.e $3\notin (2+\sqrt{-5})$.
	If ${3=(a+b\sqrt{-5})\rdot (2+\sqrt{-5})}$, then 
	\[9=N(3)=N(a+b\sqrt{-5})\rdot N(2+\sqrt{-5}) =N(a+b\sqrt{-5})\rdot 9 \implies N(a+b\sqrt{-5}) =1 \]
	which immediately tells us $b=0$ and $a=\pm 1$.\\
	But $3\ne \pm  (2+\sqrt{-5})$ hence $3\notin (2+\sqrt{-5})$ and so $2+\sqrt{-5}$ is not prime.
\end{proof}
We see that in an arbitrary integral domain, the notion of primality and irreduciblity are not the same. However, there are circumstances where they are the same thing.
\begin{prop}[title = Element in PID is prime iff it is irreducible,label =11.3]
	In a PID an element is prime \textit{iff} it is irreducible.
\end{prop}
\begin{proof}
	\hyperref[prop:11.2]{Prop 11.2} shows prime $\implies$ irred.; it remains to show irred. $\implies$ prime.\\
	Suppose $r\in R$ is irreducible and \hyperref[co:7.9]{recall that maximal ideals are prime}. Hence we will show that $(r)$ is maximal.\\
	Suppose $(r)\subset (m)\subsetneq R$, then
	\[r\in (m) \implies \exists s\in R,\, r=s\rdot m \underbrace{\implies}_{r \text{ irreducible} } s=R^\times \text{ or } m\in R^\times\]
	By assumption $(m)\subsetneq R$ and this implies
	\[m\notin R^\times \implies s\in R^\times \implies r \text{ and } m \text{ associates} \implies (r) = (m)\qedhere\]
\end{proof}
\begin{example}
	In $\Z$, the irreducibles are the primes (and their negatives)
\end{example}
\Obs that the factorization of any integer into primes is \underline{unique}!\\
We see that irreducibility, in the natural sense, is about not being able to be split up into smaller pieces (up to a unit). Primality, while similar, is more about divisibility. We saw that $9$ could be represented by fundamentally two different ideals, namely $(3)$ and $(2+\sqrt{-5})$.
\begin{dfn}[title = Unique Factorization Domain]
	A \textbf{unique factorization domain} (UFD) is an integral domain $R$ such that for all $r\in r\setminus \{0\},\, r\notin R^\times$
	\begin{enumerate}
		\item $r=p_1\rdot p_2\rdot \dots\rdot p_n$ for $p_i$ irreducible.
		\item This decomposition is unique up to associates and reordering, i.e if 
		\[r=q_1\rdot \dots \rdot q_m,\quad q_j \text{ irreducible}\]
		Then after reordering, $q_i=u_ip_i, u_i\in R^\times$ and $n=m$.
	\end{enumerate}
\end{dfn}
\begin{example}
	Fields are vacuously UFDs, because the definition of a UFD constrains non-units to certain conditions however in a field you only have units, so these constraints don't apply.
\end{example}
\begin{example}
	$\Z$ is a UFD.
\end{example}
\begin{example}
	$\Z[\sqrt{-5}]$ is \textbf{not} a UFD as
	\[3^2=9=(2+\sqrt{-5})\rdot (2-\sqrt{-5})\]
	and $3,2\pm \sqrt{-5}$ are irreducibles which are not associate.
\end{example}
\begin{prop}[title = Element in UFD is prime iff it is irreducible]
	In a UFD, an element is prime \textit{iff} it is irreducible.
\end{prop}
\begin{proof}
	It suffices to show once more that irreducible $\implies$ prime.\\
	Suppose $r\in R$ is irreducible and $a\rdot b\in (r)$ i.e there exists $c\in R$ such that $a\rdot b=r\rdot c$\\
	By unique factorization in a UFD, $a,b,c$ have unique factorizations, i.e
	\begin{align*}
	a=&p_1\rdot p_2\rdot \dots \rdot p_n,\quad p_i \text{ irreducible, unique}\\
	b=&q_1\rdot q_2\rdot \dots \rdot q_m,\quad q_j \text{ irreducible, unique}\\
	c=&r_1\rdot r_2\rdot \dots \rdot r_l,\quad r_k \text{ irreducible, unique}
	\end{align*}
	Since $a\rdot b=r\rdot c$ we have
	\[p_1\rdot p_2\rdot \dots \rdot p_n \rdot q_1 \rdot \dots \rdot q_m = r \rdot r_1 \rdot r_2\rdot  \dots \rdot r_l\]
	since $r$ is irreducible, both sides are factorizations into irreducibles. By unique factorization and w.l.o.g. 
	\[r=u\rdot p_1,\, u\in R^\times \implies \divs{r}{a}\qedhere\]
	and so $a\in (r)$ which means $(r)$ is a prime ideal (with $q$ you get $b\in (r)$).
\end{proof}
\newpage
\begin{prop}[title=Nonzero elements in UFD have GCD]
	Let $a,b\in R\setminus \{0\}$ in a UFD. Then there is a greatest common divisor of $a,b$ in $R$.
\end{prop}
\begin{proof}
	We write for $u,v\in R^\times$ and $p_i$'s irreducible
	\begin{align*}
	a=u\rdot p_1^{e_1}\rdot p_2^{e_2}\rdot \dots \rdot p_n^{e_n}\\
	b=v\rdot p_1^{f_1}\rdot p_2^{f_2}\rdot \dots \rdot p_n^{f_n}
	\end{align*}
	We allow some exponents to be $0$ ($p_i^0=1$) and we require $p_i\ne p_j$ if $i\ne j$ for example
	\[\begin{pmatrix}
	12=2^2\rdot 3 \to 12= 2^2\rdot 3^1\rdot 5^0 \\
	20=2^2\rdot 5 \to 20 =2^2\rdot 3^0\rdot 5^1 
	\end{pmatrix}\]
	So the elements $a,b$ can be written in terms of the same irreducibles to certain powers.
	\begin{claim}
		\[d=p_1^{\min\{e_1,d_1\}}\rdot p_2^{\min\{e_2,d_2\}}\rdot \dots \rdot p_n^{\min\{e_n,d_n\}}\]
		is the $\gcd$ of $a$ and $b$.
	\end{claim}
	\begin{proof}
		Clearly $\divs{d}{a},\, \divs{d}{b}$, by construction, since it's factors come from the same set of factors of $a,b$, namely $p_i$. Since $d$ is a common divisor, it remains to show it is a greatest common divisor.\\
		If $\divs{c}{a}, \, \divs{c}{b}$, then we want to see that $\divs{c}{d}$. 
		Unique factorization tells us
		\[c=q_1^{g_1}\rdot \dots \rdot q_m^{g_m}, \quad q_i \text{ irreducible},\quad  q_i\ne q_j, \text{ and } g_i>0\]
		Since $\divs{c}{a},\, \divs{c}{b}$, then after (possibly) changing associates we see that the set of factors of $c$ have to be factors of $a$ and $b$, i.e
		\[\{q_1,\dots,q_n\}\subset \{p_1,\dots,p_n\}\]
		and since $c$ divides both $a$ and $b$ then $g_i\le \min\{e_i,f_i\}$ and thus $\divs{c}{d}$.
	\end{proof}
	Therefore there exists a greatest common divisor of $a,b$ in $R$, namely, the $d$ we have shown.
\end{proof}
\end{document}