\documentclass[../Main.tex]{subfiles}
\setcounter{chapter}{2}

\begin{document}
	\chapter{L3: Ring Homomorphisms}
	\section*{Polynomial Rings}
	Fix a commutative ring $R$ with $1$ (e.g. $R=\Z, R=\Q$, etc)
	Let $X$ be an indeterminate (this means $X$ is just a symbol without an exact representation, compared to to when you think $x$ is a variable representing a number). \href{https://en.wikipedia.org/wiki/Indeterminate_(variable)}{Read more about it here!}
	\begin{dfn}[title = Polynomial in a ring]
		A \textbf{polynomial} in $X$ with coefficients in $R$ is a formal, finite sum
		\[a_nX^n+a_{n-1}X^{n-1}+\dots+a_1X+a_0, \quad a_i\in R, i\in \{0, \dots, n\}\]
		\Note If $a_n\ne 0$ and $a_m=0, \quad \forall m>n$. Then we say the \textbf{degree} of the polynomial is $n$.
		If $a_k=1$, we often omit it from the notation, e.g
		\[X^2+2\]
		has a $1$ "missing" infront of $X^2$.\\
		If $a_n=1$, we say the polynomial is \textbf{monic} 
	\end{dfn}
	\begin{dfn}[title=Ring of Polynomials and Constant Polynomial]
		The \textbf{set of polynomials} in $X$ w/ coefficients in $R$ is denoted
		\[R[X] \coloneqq \{a_nX^n+\dots+a_0|a_i \in R\} \]
		If the degree of $p\in R[X]$ is zero, we say $p$ is a \textbf{constant} polynomial.
	\end{dfn}
	\Obs that there is an obvious inclusion map from a ring into the ring of polynomials, by taking each element $a\in R$ to the constant polynomial $a\in R[X]$.
	\begin{align*}
		&R \to R[X]\\
		&a\mapsto a
	\end{align*}
	\begin{claim}
		$R[X]$ is a ring.
	\end{claim}
	\begin{proof}We check the ring properties \\
		(i) Closure under addition
		\begin{align*}
		&(a_nX^n+a_{n-1}X^{n-1}+\dots+ a_1X+a_0) + (b_nX^n+b_{n-1}X^{n-1}+\dots+b_1X+b_0) \\
		=& (a_n+b_n)X^n +(a_{n-1}+b_{n-1})X^{n_1}+\dots+(a_1+b_1)X+(a_0+b_0)
		\end{align*}
		(ii) Closure under multiplication
		\begin{align*}
			&(a_nX^n+a_{n-1}X^{n-1}+\dots+ a_1X+a_0) \rdot (b_nX^n+b_{n-1}X^{n-1}+\dots+b_1X+b_0) \\
			& \begin{aligned}=(a_0\rdot b_0)+(a_1\rdot b_0+a_0\rdot b_1)X &+ (a_2\rdot b_0 + a_1\rdot b_1+a_0\rdot b_2)X^2\\
			&+\dots+\left(\sum_{k=0}^{l}a_k\rdot b_{l-k}\right) X^l + \dots + (a_n\rdot b_m)X^{n+m} \qedhere
			\end{aligned}
		\end{align*}
	\end{proof}
	\newpage
	\begin{example}
		$\Z[X],\Q[X],\modZ{3}[X]$, which are rings of polynomials with coefficients in different number systems. In particular, we may write $\modZ{3}$ coefficients without the "overbar" notation,
		\[X+2, X^3+2X^2+1 \in \modZ{3}[X]\]
	\end{example}
	Factoring polynomials depends on the coefficient ring. For example
	\begin{align*}
		&X^2-2\in \Z[X] \\
		&X^2-2 = (X+\sqrt{2})\rdot (X-\sqrt{2}) \in \R[X]
	\end{align*}
	Here we can see that $X^2-2$ can not be factored further in the integers, but in the real numbers it can.\\
	Similarly, $X^2+1\in\Z[X], X^2+1\in \R[X]$. These polynomials doesn't factor in either ring, but it does factor in $\C[X]$
	\[X ^2+1=(X+i)(X-i)\]
	it also factors in $\modZ{2}[X]$
	\[X^2+1=(X+1)(X+1) \Mod{2}\]
	Because $X^2+2X+1\equiv X^2+1 \Mod{2}$
	\begin{prop}[title = \texorpdfstring{$R[X]$}{R[X]} is an integral domain]
		Let $R$ be an integral domain and
		$p(X),q(X)\in R[X]$
		\begin{enumerate}
			\item  $\deg(p(X)\rdot q(X))$ = $\deg p(X)+$  $\deg q(X)$.
			\item $R[X]^\times = R^\times$
			\item $R[X]$ is an integral domain
		\end{enumerate}
	\end{prop}
	\begin{proof}~\\
	(i) The leading term is
			\[(a_n\rdot b_m)X^{n+m}\]
			Since $R$ is an integral domain and $a_n,b_m\ne 0$. Then $a_n\rdot b_m\ne 0$ (This also proves (iii))
	(ii) Suppose $p(X) \in R[X]^\times$, say $p(X)\rdot q(X)=1$.\\
			Then \[\deg(p\rdot q) = \deg(1) = 0\implies \deg(p)+\deg(q)=0\implies \deg(p)=\deg(q)=0\implies p(X)\in R\]
		i.e $p(X)$ is a constant polynomial whose constant coefficient, say $p$, is from the ring $R$. Hence, since $p(X)$ is a unit, so is $p$.
	\end{proof}
	\begin{example}
		Consider $2X^2 +1, 2X^5+3X \in \modZ{4}[X]$
		\begin{align*}
		(2X^2+1)\rdot (2X^5+3X)=2\rdot 2X^7 + \text{ lower terms}
		= 0\rdot X^7 + \text{ lower terms}
		\end{align*}
	This implies
	\[\deg \left((2X^2+1)\rdot (2X^5+3X)\right)<\deg(2X^2+1)+\deg(2X^5+3x) \]
	When $R$ isn't an integral domain, the degree of the product of polynomials can be less than the degree of their sum (in general, the degree is at most the sum).
	\end{example}
\section*{Ring Homomorphisms}
\begin{dfn}[title = Ring Homomorphism and Isomorphism]
	Let $R,S$ be rings. A \textbf{ring homomorphism} is a map $f: R \to S$ such that
	\begin{enumerate}
		\item $f(a+_R b) = f(a) +_S f(b) \qquad (\textbf{Group homomorphism})$
		\item $f(a\rdot_R b) = \,\,f(a)\,\rdot_S f(b)$
	\end{enumerate}
	If $f$ is a bijective ring homomorphism, we say it is a \textbf{ring isomorphism}.\\
	We say, in this case $R$ is \textbf{isomorphic} to $S$ as rings and write
	\[R \cong S\]
\end{dfn}
\begin{dfn}[title = Kernel]
	The \textbf{kernel} of a ring homomorphism $f: R\to S$ is the subset
	\[\Ker f \coloneqq f^{-1}(0_S)\subset R\]
\end{dfn}
\begin{prop}[title = \texorpdfstring{$\Ker f$}{Ker f} is an ideal]
	Let $R,S$ be rings and $f: R \to S$ a homomorphism, then
	\begin{enumerate}
		\item $\Img f \subset S$ is a subring
		\item $\Ker f \subset R$ is a subring
	\end{enumerate}
	where $\Img$ is the image of $f$ and $\Ker$ is the kernel.\\
	Moreover, if $r\in R$, $a\in \Ker f$ then $r\rdot a \in \Ker f$.\\
	 (this is a stronger property of the kernel, which shows it is more than just a subring, since it is also closed under multiplication with elements from outside the kernel, in particular from the ring).
\end{prop}
Both proofs rely on using the Subring Criterion Test mentioned in Lecture 2.
\begin{proof}[Proof (i)]First, we check show that it is non empty.
	 \begin{claim}
		$f(0_R) = 0_S$ and in particular $\Img f \ne \emptyset$.
		\begin{proof}
			By definition of ring homomorphism
			\[f(0_R)=f(0_R+0_R)=f(0_R)+f(0_R) \implies 0_s = f(0_R)\]
			Where we have subtracted (in $S$) $f(0_R)$ from both sides.
		\end{proof}
	\end{claim}
	Suppose now $f(a), f(b) \in \Img f$, then
	\[f(a)\rdot f(b) = f(a\rdot b) \in \Img f\]
	which shows the product is also in the image.\\
	Finally, whats left to show is that the difference is also in the image.
	To see $f(a)-f(b) \in \Img f$, it suffices to see that $-f(b)=f(-b)$.\\
	\begin{claim}
		$-f(b)=f(-b)$
	\end{claim}
	\begin{proof}
		Again using the ring homomorphism definition
		\[0=f(0_R)=f(b+(-b))=f(b)+f(-b) \implies f(-b)=-f(b)\qedhere\]
	\end{proof}
	Therefore, with the subring criterion satisfied, then $\Img f$ is a subring in $S$.
	\end{proof}
	\begin{proof}[Proof (ii)]
	Since $f(0_R)=0_S \implies 0_R \in \Ker f$, hence $\Ker f$ is nonempty.\\
	Suppose $a,b\in \Ker f$, then
	\[f(a-b) = f(a)-f(b)=0-0=0 \implies a-b \in \Ker f\]
	and
	\[f(a\rdot b) = f(a)\rdot f(b) = 0\rdot 0 = 0 \implies a\rdot b \in \Ker f\]
	Hence, $\Ker f$ is a subring in $R$.\\
	Now suppose $r\in R$
	\[f(r\rdot a)=f(r)\rdot f(a)=f(r)\rdot 0=0\]
	which proves the additional property.
\end{proof}
\begin{example}
	Consider the map which takes even numbers to $0$ and odd numbers to $1$.
	\begin{align*}
		f: \,&\Z \to \modZ{2}\\
		&a \mapsto a \Mod{2}
	\end{align*}
	Check the possible situations (these show the sums and products follow the homomorphism properties)
	\begin{equation*}
	\text{Addition}  \quad \left|\quad\begin{array}{>$l<$}
	$\obar{0}+ \obar{0} = \obar{0}$ \quad
	even $+ $ even = even\\
	$\obar{0}+ \obar{1}= \obar{1}$ \quad
	even $+ $ odd = odd\\
	$\obar{1}+ \obar{1} = \obar{0}$\quad
	odd $+ $ odd = even
	\end{array}\right.
	\end{equation*}
	\begin{equation*}
	\text{Multiplication} \quad \left|\quad\begin{array}{>$l<$}
	$\obar{0}\rdot \obar{0} = \obar{0}$ \quad
	even $\rdot $ even = even\\
	$\obar{0}\rdot \obar{1}= \obar{0}$ \quad
	even $\rdot $ odd = even\\
	$\obar{1}\rdot \obar{1} = \obar{1}$\quad
	odd $\rdot $ odd = odd
	\end{array}\right.
	\end{equation*}
	Therefore $\Ker f = \{\text{evens}\} = 2\Z$ and observe that
	\[f^{-1}(\obar{1}) = \{ \text{odds}\} = 1+2\Z = \{1+2n | n\in Z\} = 1+\Ker f\]
\end{example}
\begin{example}
	The following is a non-example. Consider
	\begin{align*}
		f_n:\,& \Z \to \Z \\
		&a \mapsto n\rdot a
	\end{align*}
	Then
	\[f_n(a+b)=n\rdot (a+b)=n\rdot a+n\rdot b= f_n(a)+f_n(b)\]
	But 
	\[f_n(a\rdot b)=n(a\rdot b) \stackrel{?}{=} n^2(a\rdot b)=(n\rdot a)\rdot (n\rdot b)=f_n(a)\rdot f_n(b)\]
	So $f_n$ is a ring homomorphism if and only if $n^2=n$ (i.e $n=0,1$). $f_0$ is the constant map zero and $f_1$ is the identity.\\
	Therefore $f_2,f_3,\dots $ are \textbf{NOT} ring homomorphisms. In particular, it shows that a group homomorphism is not necessarily a ring homomorphism.
\end{example}
\begin{example}
	Here is a polynomial homomorphism which maps a polynomial to its own constant term
	\begin{align*}
		\phi: &\R[X] \to \R \\
		&p(X) \mapsto p(0)
	\end{align*}
	This can easily be checked
	\begin{align*}
		&\phi(p+q) = (p+q)(0)=p(0)+q(0)=\phi(p)\phi(q)\\
		&\phi(p\rdot q)=(p\rdot q)(0)=p(0)\rdot q(0)=\phi(p)\rdot \phi(q)
	\end{align*}
	Its kernel (which are polynomials who have $0$ as a root) can be written
	\[\Ker \{p\in \R[X]\,|\,p(0)=0\} = \{p\in R[X]\,|\,p(X) = X\rdot p'(X) \text{ for some } p'\in \R[X]\}\]
	($p'$ is not the derivative, just another polynomial).
\end{example}
\textbf{Question}: What about
\begin{align*}
	\phi_1 \,: &\R[X] \to \R\\
	&p(X) \mapsto p(1)
\end{align*}
\end{document}