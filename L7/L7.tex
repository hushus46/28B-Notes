\documentclass[../Main.tex]{subfiles}
\setcounter{chapter}{6}

\begin{document}
\chapter{L7: Maximal Ideals}
Let $R$ be a commutative ring with $1\ne 0$.
\begin{prop}[title = Ideals containing units]
	Let $I\subset R$ an ideal
	\begin{enumerate}
		\item $I=R$ if and only if $I$ contains a unit.
		\item $R$ is a field if and only if the only ideals of $R$ and $0$ and $R$
	\end{enumerate}
\end{prop}
\begin{proof}~\\
	(i) If $I=R$, then $1\in I$\\
	Conversely, if $u\in I$ and $u\in R^\times$ say $u\rdot v=1$, then $u\rdot v=1\in I$ implies, if $r \in R$, then 
	\[r\rdot (u\rdot v)=r\in I \implies R\subset I \implies R=I\]
	(ii) If $I\subset R$ is an ideal in a field, and $\exists a \in I \setminus \{0\}$ (non-zero element of the field), then $a\in R^\times$ (since it is a field) implies $I=R$ (by part (i)).\\
	Conversely, suppose $0$ and $R$ are the only ideals in $R$. Let $a\in R\setminus \{0\}$ and consider $(a)\subset R$, then
	\[(a)\ne 0 \implies (a)=R \underbrace{\implies}_{\text{by part (i)}} \,\,\exists u\in (a), \, u \in R^\times (\text{say } u\rdot v=1)\]
	Since $u\in (a)$, we may write $u=r\rdot a, r\in R$, then
	\[(r\rdot a)\rdot v=u\rdot v=1=a\rdot (r\rdot v)\implies a\in R^\times \implies R \text{ is a field}\]
\end{proof}
\begin{crl}[title = Homomorphism from field to ring is injective]
	If $F$ is a field, then any nonzero ring homomorphism
	\[f: F\to R\]
	is an injective map
\end{crl}
\begin{proof}
	$\Ker f =0$ or $F$. Because $f$ is nonzero, we conclude that $\Ker f =0$, which means $f$ is injective since the only element that maps to $0$ is $0$.
\end{proof}
\begin{dfn}[title = Maximal Ideal]
	An ideal $M\subset R$ is called a \textbf{maximal ideal} if
	\begin{enumerate}
		\item $M\ne R$
		\item If $I\subset R$ is an ideal such that $M\subset I$, then $I=M$ or $I=R$
	\end{enumerate}
\end{dfn}
Not all rings admit maximal ideals and a given ring may admit multiple maximal ideals, e.g $2\Z,3\Z$ are maximal ideals in $\Z$.
\newpage
\section*{A Digression on Zorn's Lemma}
\begin{dfn}[title = Partial Order]
	A \textbf{partial order} on a non-empty set $A$ is a relation $\le$ such that
	\begin{enumerate}
		\item $x\le x$ (Reflexive)
		\item $x\le y, y\le x \implies x =y$ (Anti-symmetric)
		\item $x\le y, y\le z\implies x\le z$ (Transitive)
	\end{enumerate}
\end{dfn}
\begin{example}~\\
	If $X$ is any set then the power set (the set of all subsets) is written
	\[\wp(X) = \{\text{subsets } U \subset X\}\]
	Then inclusion is a partial order on $\wp(X)$, e.g \[
	\begin{tikzcd}
	& \{a,b,c\} \\
	\{a,b\}\arrow[ur, phantom, "\subset"{sloped}] & \{a,c\}\arrow[u, phantom, "\subset"{sloped}] & \{b,c\}\arrow[ul, phantom, "\supset"{sloped}] \\
	\{a\}\arrow[u, phantom, "\subset"{sloped}]\arrow[ur, phantom, "\subset"{sloped}, pos=.4] & \{b\}\arrow[ul, phantom, "\supset"{sloped}, pos=.4]\arrow[ur, phantom, "\subset"{sloped}, pos=.4] & \{c\}\arrow[ul, phantom, "\supset"{sloped}, pos=.4]\arrow[u, phantom, "\subset"{sloped}] \\
	& \varnothing\arrow[ul, phantom, "\supset"{sloped}]\arrow[u, phantom, "\subset"{sloped}]\arrow[ur, phantom, "\subset"{sloped}]
	\end{tikzcd}
	\]
\end{example}
\begin{dfn}[title = {Poset, Chain, Upper Bound, Maximal Element}]
	If $A,\le$ is a \textbf{partially ordered set} (\textbf{poset}), then
	\begin{enumerate}
		\item A subset $B\subset A$ is a \textbf{chain} if $\forall x,y \in B \implies x\le y$ or $y\le x$ (everything can be compared).
		\item An \textbf{upper bound} on a subset $B\subset A$ is an element $u\in A$ such that \[\forall b\in B, b\le u\]
		\item A \textbf{maximal element} of a subset $B\subset A$ is an element of $m\in B$ such that if $b\in B$ and $b\ge m$, then $b=m$.
	\end{enumerate}
\end{dfn}
\begin{lem}[title = Zorn's Lemma]
	If $A$ is a non-empty poset such that every chain admits an upper bound, then $A$ has a maximal element.
\end{lem}~\\
\hrule
\begin{prop}[title = All proper ideals contained in maximal ideal]
	If $R$ is a commutative ring with $1\ne 0$, then every proper ideal is contained in a maximal ideal
\end{prop}
\begin{proof}~\\
	Let $I \subsetneq R$ be a proper ideal.\\
	Consider
	\[\mathcal{S}\coloneqq \{\text{proper ideals of } R \text{ containing } I\} \]
	$\mathcal{S}$ is partially ordered by inclusion
	\begin{tikzcd}[arrows=dash]
		&R \ar[dl]\ar[dr]\\
		I_1\ar[dr] && I_2\ar[dl]\ar[dr] \\
		&I_3\ar[d] && I_4\\
		&I \ar[urr]
	\end{tikzcd}\\
	A chain of ideals in $\mathcal{S}$ is a collection of ideals
	\[\mathcal{C}=\{\dots\subset I_{-1} \subset I_0\subset I_1\subset I_2 \subset \dots\}\]
	and to apply Zorn's Lemma, we need to show $\mathcal{C}$ has an upper bound.\\
	Let \[J=\bigcup\limits_{I_k\in C}I_k\]
	\begin{claim}
	$J$ is an ideal containing $I$.
	\end{claim}
	\begin{proof}~\\
		$I\subset J$ is clear, since $I$ is contained in all the ideals $I_k \in S$. It remains to show $J$ itself is an ideal.\\
		$0\in J$ because $0\in I_k$ for any $k$.\\
		If $a,b\in J$, then $\exists I_{k_1}, I_{k_2}$ such that $a\in I_{k_1}, b\in I_{k_2}$, so w.l.o.g say $I_{k_1}\subset I_{k_2}$, then
		\[a,b\in I_{k_2}\implies a-b \in I_{k_2}\subset J \implies a-b\in J\]
		If $r \in R$, then $r\rdot a\in I_{k_2}\subset J\implies r\rdot a \in J$.\\
		Hence, $J$ is an ideal containing $I$.
	\end{proof}
	Therefore $J$ is an upper bound for $\mathcal{C}$ and we can apply Zorn's lemma. \\Therefore, $\mathcal{S}$ admits a maximal element, i.e a proper ideal $M\subset R$ such that $I\subset M$. \\If $M'\subset R$ is an ideal such that $M\subset M'$, then $I\subset M'$ and so
	\[\underbrace{M'\in \mathcal{S}}_{M' \text{ is proper}} \implies M' = M
	\quad \text{ or }
	\underbrace{M'\notin \mathcal{S}}_{M' \text{ is not proper}} \implies M'=R\]
\end{proof}
\begin{thm}[title = \texorpdfstring{$M$}{M} maximal in comm. \texorpdfstring{$R \Longleftrightarrow R/M$}{R iff R/M} is field]
	If $R$ is a commutative ring with $1\ne 0$, then $M\subset R$ is maximal if and only if $R/M$ is a field.
\end{thm}

\begin{proof}~\\
	Using the Lattice (fourth) Isomorphism Theorem  we have
	\begin{align*}
	\{\text{Ideals of } R \text{ containing } M \} &\longleftrightarrow \{\text{Ideals of } R/M\}\\
	\{M,R\} &\longleftrightarrow \{0, R/M\}
	\end{align*}
	Since, the only ideals of $R/M$ are $0$ and itself, $R/M$ is a field by Prop 7.1 (ii).
\end{proof}
Recall: $P\subset R$ is prime if and only if $R/P$ is an integral domain.
\begin{crl}[title= Maximal ideals are prime]
	Maximal ideals are prime.
\end{crl}
\begin{proof}~\\
	If $M$ is maximal then $R/M$ is a field. Therefore, $R/M$ is an integral domain and hence $M$ is prime.
\end{proof}
\begin{example}~\\
	$n\Z\subset \Z$ is maximal if and only if $\modZ{n}$ is a field, i.e $n$ is prime.\\
	So in $\Z$ we have
	\[\{\text{prime ideals}\} =\{\text{maximal ideals}\} \]
\end{example}
\begin{example}~\\
	The ideal generated by $x$, $(x) \subset \Z[x]$ is prime (check).\\
	However, it is not maximal as $(x) \subset (2,x)$, but $1\notin (2,x)$ and therefore $(2,x) \subsetneq \Z[x]$.
	So, in this case prime ideals are not necessarily maximal.
\end{example}
\begin{example}~\\
	$(x)\subset \R[x]$ is maximal.
	\[\R[x]/(x) \cong \R\]
	and recall $\R$ is a field.
\end{example}
\end{document}