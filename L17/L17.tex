\documentclass[../Main.tex]{subfiles}
\setcounter{chapter}{16}

\begin{document}
\phantomsection
\chapter{L17: Spanning sets and free modules}
\begin{dfn}
	Let $M$ be an $R$-module.\\
	An \textbf{$R$-linear combination} of elements $m_1,\dots,m_n\in M$ is an element of the form
	\[a_1\rdot m_1+a_2\rdot m_2+\dots+a_n\rdot m_n\quad a_i\in R\]
	We say a subset $A\subset M$ \textbf{spans} or \textbf{generates} the module if every element of $M$ is an $R$-linear combination of elements in $A$.\\
	More generally, if $B\subset M$, the \textbf{submodule spanned/generated by} $B$ is
	\[RB \coloneqq \{a_1\rdot m_1+a_2\rdot m_2+\dots+a_n\rdot m_n\mid n\in \Z^+,\, a_i\in R,\, m_i \in B\}\]
\end{dfn}
\Exr Show that $RB$ is an $R$-module
\begin{example}
	For any ring $R$ with $1\ne 0$ every element is a "linear combination" of $\{1\}$ i.e. if $r\in R$, then $r=r\rdot 1$.\\
	So $R=R\{1\}$ is spanned by a single element as an $R$-module
\end{example}
\begin{example}
	The polynomial ring $R[X]$ has a natural $R$-module structure:\\
	If $a\in R, p(X)=a_0+a_1X+\dots+a_nX^n\in R[X]$ then 
	\[a\rdot \left(a_0+a_1X+\dots+a_nX^n\right) \coloneqq (a\rdot a_0)+(a_\rdot a_1)\rdot X+\dots+(a\rdot a_m)X^n\]
	$R[X]$ is spanned by $\{1,X,X^2,X^3,X^4,\dots\}$
\end{example}
\Obs $R[X]$ has \textbf{no} finite spanning set!
To see this, suppose $R[X]$ is spanned by
\[p_1(X),p_2(X),\dots,p_n(X)\in R[X]\]
Let $d=\max\{\deg p_1(X),\dots, \deg p_n(X)\}$
Then $d< \infty \implies \forall a_1,\dots,a_n \in R$
\[\deg[a_1\rdot p_1(X)+a_2\rdot p_2(X)+\dots+a_n\rdot p_n(X)]\le d \implies X^{d+1} \notin \Span\{p_1(X),\dots,p_n(X)\}\]
\begin{dfn}
	We say an $R$-module $M$ is \textbf{finitely generated} if it is has a finite spanning set. We say $M$ is \textbf{cyclic} if it is spanned by a single element.
\end{dfn}
\begin{example}
	If $R$ is a ring, $A\subset R$. Then $RA = (A)$ (the module generated by $A$ is the ideal generated by $A$). A cyclic submodule of $R$ is just a principal ideal.
\end{example}
\begin{example}
	$R$ a ring, $F=R^n$ is the free $R$-module of rank $n$. $F$ has a natural spanning set:
	\[E_n \coloneqq \begin{Bmatrix}
		e_1=(1,0,0,\dots,0)\\
		e_2=(0,1,0,\dots,0)\\
		e_3=(0,0,1,\dots,0)\\
		\dots\\
		e_n=(0,0,0,\dots,0,1)
	\end{Bmatrix}\]
	Any element $(a_1,a_2,\dots,a_n)\in R^n$ can be written as
	\begin{align*}
		(a_1,a_2,\dots,a_n)=&a_1\rdot (1,0,0,\dots,0)+a_2\rdot (0,1,0,\dots,0) + \dots + a_n \rdot (0,0,0,\dots,1)\\
		=& a_1\rdot e_1+a_2\rdot e_2+\dots+a_n\rdot e_n
	\end{align*}
\end{example}
\section*{Recontextualizing the free $R$-module of rank $n$:}
Consider the set $\{1,2,3,\dots,n\}$ 
A function 
\begin{align*}
a\colon \{1,2,3,\dots,n\} &\to R\\
1&\mapsto a(1)=a_1\\
2&\mapsto a(2)=a_2\\
\dots\\
n&\mapsto a(n)=a_n
\end{align*}
we can think of an ordered $n$-tuple of elements in $R$ as a function
\[a\colon\{1,2,\dots,n\}\to R\]
i.e. we can think of $R^n$ as
\[R^n=\{a\colon \{1,2,\dots,n\}\to R\]
The obvious addition is
\begin{align*}
a+b\colon \{1,2,\dots,n\} &\to R\\
1&\mapsto a(1)+b(1)\\
2&\mapsto a(2)+b(2)\\
\dots\\
n&\mapsto a(n)+b(n)
\end{align*}
The obvious scalar multiplication is
\begin{align*}
r\rdot a\colon \{1,2,\dots,n\} &\to R\\
1&\mapsto r\rdot a(1)\\
2&\mapsto r\rdot a(2)\\
\dots\\
n&\mapsto r\rdot a(n)
\end{align*}
\begin{dfn}
Fix a ring $R$. An $R$-module $F$ is \textbf{free} on a set $A$ if $\forall m\in F$ there are \textbf{unique} elements 
\begin{align*}
m_1,m_2,\dots,m_n\in A\\
a_1,a_2,\dots,a_n\in R
\end{align*}
s.t. $m=a_1\rdot m_1+a_2\rdot m_2+\dots+a_n\rdot m_n$.\\
We call $A$ set of \textbf{free generators} of $F$ or a \textbf{basis} of $F$.\\
\Note Usually, we ask that the basis is \textbf{ordered} in some way.
\end{dfn}
\begin{example}
	The set $E_n=\{e_1,e_2,\dots,e_n\}$ is a basis for the free module of rank $n$.
\end{example}
\begin{example}
	$\modZ{2}$ is a non-free $Z$-module.
	\begin{align*}
	\obar{1}&=1\rdot \obar{1}\\
	&=3\rdot \obar{1}
	\end{align*}
\end{example}
\begin{example}
	Is every submodule of a free module free?\\
	$\modZ{4}$ is a free module over $\modZ{4}$\\
	\Exr Check that $\modZ{4} = \modZ{4}\{\obar{1}\}$ is free.\\
	$2\modZ{4}=\{\obar{0},\obar{2}\} \subset \modZ{4}$ is a submodule.\\
	BUT:
	\begin{align*}
	\obar{2}\rdot \obar{2}=\obar{0}\\
	\obar{0}\rdot \obar{2}=\obar{0}
	\end{align*}
	There is no unique way of writing $\obar{0}$ as a $(\modZ{4})$-linear combination of $\{\obar{2}\}$. This implies $2\rdot \modZ{4}=(\obar{2})$ is \textbf{not} free
\end{example}
\begin{example}
	Fix a ring $R$. Let $A$ be \textbf{any} set
	\[F_R(A) \coloneqq \{\phi\colon A\to R\mid \phi(a)=0 \text{ for all but finitely many }a\in A\}\]
\end{example}
\begin{prop}
	$F_R(A)$ is a free module over $R$ on the set $A$.
\end{prop}
\begin{proof}
	Let $\phi,\psi\colon A\to R$ then addition
	\begin{align*}
		\phi+\psi\colon A&\to R &   r\rdot \phi \colon A&\to R\\
		a&\mapsto \phi(a)+\psi(a)& a&\mapsto r\rdot \phi(a)
	\end{align*}
	Consider the inclusion map
	\begin{align*}
	\iota\colon A &\to F_R(A)\\
	a&\mapsto 
	\begin{aligned}[t]
	\phi_a\colon A&\to R\\
	x&\mapsto \begin{cases}
	1&x=a\\
	0&x\ne a
	\end{cases}
	\end{aligned}
	\end{align*}
	Obviously this map is injective. If $\phi_a=\phi_b$ then $\phi_a(a)=1=\phi_b(a)\implies a=b$.\\
	We call $\iota(A)=E_A$ and we see that\\
	(1) $E_A$ spans $F_R(A)$
	\begin{proof}
		$(\phi\colon A\to R)\in F_R(A)$\\
		Let $\{a_1,\dots,a_n\}\subset A$ such that $\phi(a_i)\ne 0$. Then
		\begin{align*}
		&\phi(a_i)=\phi(a_i)\rdot 1 = \phi(a_i)\rdot \phi_{a_i}(a_i)\\
		\implies & \phi \equiv \underbrace{\phi(a_1)}_{\in R}\rdot \phi_{a_1} +\underbrace{\phi(a_2)}_{\in R}\rdot \phi_{a_2}+\dots+\underbrace{\phi(a_n)}_{\in R}\rdot \phi_{a_n}\\
		&\implies \phi \in \Span E_a \qedhere
		\end{align*}
	\end{proof}
	(2) $F_R(A)$ is free on $E_A$
	\begin{proof}
		Suppose
		\begin{align*}
		\phi=& r_1\rdot \phi_{a_1}+r_2\rdot \phi_{a_2}+\dots+r_n\rdot \phi_{a_n}\\
		&=s_1\rdot \phi_{a_1}+s_2\rdot \phi_{a_2}+\dots+s_n\rdot \phi_{a_n}
		\end{align*}
		Then
		\begin{align*}
		&(r_1-s_1)\rdot \phi_{a_1}+(r_2-s_2)\rdot \phi_{a_2}+\dots+(r_n-s_n)\rdot \phi_{a_n}=0\\
		\implies & (r_1-s_1)\underbrace{\rdot \phi_{a_1}(a_1)}_{=1}+(r_2-s_2)\rdot \cancelto{0}{\phi_{a_2}(a_1)}+\dots+(r_n-s_n)\rdot \cancelto{0}{\phi_{a_n}(a_1)}=0\\
		\implies & (r_1-s_1)\rdot 1=(r_1-s_1)=0\implies r_1=s_1
		\end{align*}
		Similarly $r_i=s_i\,\forall i$
	\end{proof}
\end{proof}
\begin{thm}
	Let $R$ be a ring, $A$ is any set, $M$ is an $R$-module such that there exists $f\colon A\to M$.\\
	There is a unique $R$-module homomorphism
	\[\Phi_A\colon F(A)\to A\]
	such that
	\[\text{TIKZ}\]
\end{thm}
\begin{proof}
	\begin{align*}
	\Phi_A\colon F(A)&\to M\\
	(\phi\colon A\to R)&\mapsto \sum_{a\in A}\underbrace{\phi(a)}_{\in R}\rdot \underbrace{f(a)}_{\in M}
	\end{align*}
\end{proof}
\begin{crl}
	If $R$ is a ring and $F$ is any free module on a set $A$, then $F\cong F(A)$
\end{crl}
\begin{proof}
	$A\subset F$ that generates $F$ freely over $R$, $j\colon A\to F$
	\begin{center}
	\begin{tikzcd}[scale cd =1.2]
		A \arrow{r}{\iota} \arrow[swap]{dr}{j} & F(A)\arrow{d}{\Phi_A} \\
		& F
	\end{tikzcd}
	\end{center}
	There is an obvious map \begin{align*}
	\Psi_A\colon F&\to F(A)\\
	r_1a_1+\dots+r_na_n &\mapsto r_1\phi_{a_1}+r_2\phi_{a_2}+\dots+r_n\phi_{a_n}
	\end{align*}
	Clearly this map
	\begin{center}
		\begin{tikzcd}[scale cd =1.2]
			A \arrow{r}{\iota} \arrow{dr}{j}\arrow[swap]{ddr}{\iota}  & F(A)\arrow{d}{\Phi_A}\arrow[bend left=60]{dd}{\Id_{F(A)}} \\
			& F\arrow{d}{\Psi_A}\\
			&F(A)
		\end{tikzcd}
	\end{center}
	By uniqueness $\Psi_A\circ \Phi_A=\Id_{F(A)}$ and hence $\Phi_A\colon F(A)\to F$ is an $R$-module isomorphism
\end{proof}
\end{document}