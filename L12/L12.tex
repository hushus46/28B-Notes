\documentclass[../Main.tex]{subfiles}
\setcounter{chapter}{11}

\begin{document}
\chapter{L12: PIDs are UFDs and Polynomial Rings}
\begin{dfn}[title = {Ascending Chains, Noetherian Ring}]
	Let $R$ be a commutative ring with $1\ne 0$.\\
	An \textbf{ascending chain} of ideals in $R$ is a sequence
	\[I_1\subset I_2\subset I_3\subset\dots.\subset R\]
	We say an ascending chain \textbf{stabilizes} if there exists $N\in \N$ such that for all $n,m \ge N$, $I_n=I_m$.\\
	We say $R$ satisfies the \textbf{ascending chain condition} (a.c.c.) if every ascending chain stabilizes.\\
	If $R$ satisfies the a.c.c., we say it is a \textbf{Noetherian ring}.
\end{dfn}
\begin{thm}[title = PID is Noetherian]
	If $R$ is a PID, then $R$ is Noetherian.
\end{thm}
\begin{proof}
	Let
	\[I_1\subset I_2\subset I_3\subset\dots.\subset R\]
	be an ascending chain in a PID.\\
	Consider
	\[I\coloneqq \bigcup\limits_{n\in \N} I_n\]
	which is an ideal. Then since $R$ is a PID, $I=(a)$ for some $a\in R$.\\
	In particular,
	\[a\in I =\bigcup\limits_{n\in \N} I_n\implies a\in I_N \]
	for some $N\in \N$.\\
	But if $a\in I_N$ then we also know $(a)\subset I_N$ implying $I\subset I_N$. \\
	But by the definition of $I$, we also have the containment in the other direction, i.e $I_N \subset I$, and hence we have $I=I_N$.\\
	Furthermore the chain stops "growing" at a finite ideal $I_N$ and so the ascending chain stabilizes
	\[I=I_N=I_{N+1}=I_{N+2}=\dots\]
	Therefore, $R$ is a Noetherian ring.
\end{proof}
\begin{thm}[title = PID is UFD]
	Every PID is a UFD.
\end{thm}
Let $R$ be a PID.\\
We want to show if $r\in R\setminus \{0\},\, r\notin R^\times$,
then $r$ admits a \textbf{unique} expression as a product of irreducibles.
\newpage
\begin{lem}[title = Existence of product of irreducibles in PID]
	A element $r$ in a PID has \textbf{some} expression as a product of irreducibles
\end{lem}
\begin{proof}~\\
	If $r$ is irreducible, then $r=r$ and we are done.\\
	If not, then we can write $r=r_1\rdot r_2,$ where $r_1,r_2\notin R^\times$. Then $r\in (r_1)$ but $(r)\ne (r_1)$ because in order for that to be the case, $r_2$ would have to a be a unit. Therefore, it is a proper subset i.e, $(r)\subsetneq (r_1)$.\\
	If $r_1,r_2$ are irreducibles, then we are done.\\
	If not, 
	\begin{align*}
	r_1=r_{11}\rdot r_{12}\\
	r_2=r_{21}\rdot r_{22}
	\end{align*}
	where $r_{ij}\notin R^\times,\, i,j\in \{1,2\}$. Again, $r_1\in (r_{11})$ but (since $r_{12}$ is not a unit)  $(r_1)\ne (r_{11})$, and hence $(r)\subsetneq (r_1)\subsetneq (r_{11})$.\\
	Since $R$ is a PID, it is also Noetherian, and so this chain stabilizes eventually. This means we will reach a point where and $(r_{1111}) =(r_{11111})$ implying $r_{1111}=r_{11111}\rdot u$ for some unit $u$, and thus $r_{1111}$ is irreducible. Hence in general $r$ will be factored into something like
	\[r=(r_{111\dots1}\rdot r_{111\dots 2})\rdot \dots \rdot (r_{222\dots1}\rdot r_{222\dots 2})\]
	where each term on the right side of the inequality is irreducible.
\end{proof}
\begin{lem}[title = Uniqueness of product of irreducibles in PID ]
	The factorization into irreducibles is \textbf{unique} (up to reordering and associates).
\end{lem}
\begin{proof}~\\
	Say the factorization into irreducibles is $r=p_1\rdot p_2\rdot \dots\rdot p_n$. We proceed by induction on $n$.\\
	\textbf{\underline{Base Case:}} If $n=1$, then $r=\underbrace{p_1}_{\text{irred.}}$ implies $r$ is irreducible.\\
	Suppose now $r$ factors into a different product of irreducibles,
	\[r=q_1\rdot q_2\rdot \dots\rdot q_n,\, n\ge 2,\, q_i \text{ irreducible } \forall i\in \{1,\dots,n\}\]
	But then $q_1,(q_2\rdot \dots\rdot  q_n) \notin R^\times$ (since by definition irreducibles are non-units) implying $r$ is not irreducible, which is a contradiction.\\
	Therefore, $r=p_1$ is the unique way to write $r$ as the product of irreducibles when $n=1$.\\
	\textbf{\underline{Induction Hypothesis:}} Now suppose if $r$ admits a factorization into at most $n-1$ irreducibles, then the factorization is unique.\\
	\textbf{\underline{Inductive Step:}} If we can write into two different factorizations
	\begin{align*}
	r&=p_1\rdot p_2\rdot \dots \rdot p_n, \quad p_i\text{'s irreducible}\\
	&=q_1\rdot q_2\rdot \dots\rdot q_m,\quad q_j\text{'s irreducible and } m\ge n
	\end{align*}
	Then $\divs{p_1}{q_1 \rdot (q_2\rdot \dots\rdot  q_m)}$ and recall \hyperref[prop:11.3]{irreducibles are prime in a PID}. Since $p_1$ is irreducible, it is prime and so either $\divs{p_1}{q_1}$ or $\divs{p_1}{(q_2\rdot \dots\rdot  q_n)}$. W.l.o.g. assume $\divs{p_1}{q_1}$ i.e $q_1=u\rdot p_1,\, u\in R$.\\
	Since $q_1$ is irreducible, then $u\in R^\times $ or $p_1 \in R^\times$. But $p_1$ is irreducible, so it can be not an element of $R^\times$, therefore $u\in R^\times$ and so $p_1$ and $q_1$ associate.\\
	So we write
	\begin{align*}r=p_1\rdot p_2\rdot \dots \rdot p_n&= q_1\rdot q_2 \rdot  \dots \rdot q_m\\
	&=(u\rdot p_1)\rdot q_2\rdot \dots \rdot q_m \end{align*}
	Since $R$ is an integral domain, we \hyperref[prop:cancel]{can cancel} $p_1$ from both sides to get
	\[\underbrace{p_2\rdot \dots \rdot p_n}_{\text{product of (n-1) irred.}} =(u\rdot q_2)\rdot q_3\rdot \dots \rdot q_m \]
	Now by our induction hypothesis, $r$ admits a unique factorization into at most $(n-1)$ irreducibles, which implies the list of irreducibles on the left and right side of the equality are the same (up to associates) i.e.
	\[\{(u\rdot q_2),q_3,q_4,\dots,q_m\}=\{p_2,p_3,\dots,p_n\}\]
	and so $m=n$ and the $p_i$'s are unique.
\end{proof}
We can now see a hierarchy for the specific structures we have discussed thus far

\begin{tikzcd}[column sep=0em,arrows=dash]
	&\text{Rings}\arrow[dl]\arrow[dr]&& \\
	\text{Non-Comm. Rings}&&\text{Comm. Rings w/}1\ne 0 \arrow[dl]\arrow[dr]&\\
	&\text{Noetherian Rings}\arrow[d]&&\text{Integral Domains}\arrow[dl]\\
	&\text{PIDs}\arrow[d]&\text{UFDs}\arrow[l]\\
	&\text{Euclidean Domain}\arrow[d]&&\\
	&\text{Fields}&&
\end{tikzcd}
\section*{Polynomial Rings (Again)}
Let $R$ be an commutative ring with $1\ne 0$.\\
Now assume $R$ is an integral domain and recall some facts we've already proven about :\\
(1) $R[X]$ is an integral domain.\\
(2) $R[X]^\times=R^\times$ e.g. $\Z[X]$, the only units are $\{\pm 1\}$.\\
(3) $\deg[p(X)\rdot q(X)]=\deg p(X)+ \deg q(X)$\\
(4) The field of fractions of $R[X]$ is the field of rational functions
	\[R(X)\coloneqq \left\{ \left. \frac{p(X)}{q(X)} \right| \, p,q\in R[X],\, q\ne 0\right\}\]
(5) If $F$ is a field, then $F[X]$ is a Euclidean Domain.
	\begin{crl}[title = {\texorpdfstring{$F[X]$}{F[X]} is PID, UFD, and Noetherian}]
		If $F$ is a field, $F[X]$ is a PID, UFD, and Noetherian.
	\end{crl}
(6) Let $I\subset R$ be an ideal and $R$ a commutative ring (not necessarily integral) and consider the ideal generated by $I$ in $R[X]$, i.e.
	\[(I)\coloneqq I[X] \coloneqq \{p(X)\in R[X]\mid \text{coeffs. are in } I\}\]
	Then 
	\[R[X]/(I)\cong (R/I)[X]\]
\begin{proof}~\\
	Consider the map
	\begin{align*}
	\phi\,\colon R[X] &\to (R/I)[X] \\
	a_0+a_1X+\dots+a_nX^n &\mapsto \obar{a_0}+\obar{a_1}X+\obar{a_2}X^2+\dots+\obar{a_n}X^n
	\end{align*}
	for example
	\begin{align*}
	\phi\,\colon \Z[X] &\to (\Z/3\Z)[X] \\
	1+2X+4X^3 &\mapsto \obar{1}+\obar{2}X+\obar{4}X^3=\obar{1}+\obar{2}X+X^3
	\end{align*}
	"Clearly" $\phi$ is a surjective ring homomorphism, so
	\[(R/I)[X]\cong R[X]/\Ker \phi\]
	But the kernel is exactly the set of polynomials with coefficients that are zero (i.e. in the ideal), hence $\Ker\phi\coloneqq\{a_0+a_1X+\dots+a_nX^n\mid a_i\in I\}=(I)$.
\end{proof}
\begin{example}
	Consider $3\Z \coloneqq \{0,3,-3,6,-6,\dots\}$ and
	\[(3\Z)\coloneqq \{a_0+a_1X+a_2X^2+\dots+a_nX^n\mid a_i\in 3\Z\} \implies \Z[X]/(3\Z)\cong (\Z/3\Z)[X]\]
	e.g $1+2X+4X^3=1+2X+X^3+\underbrace{3X^3}_{\in (3\Z)}$; so we can think about the coefficients in either ring
	\[\underbrace{1,2,4}_{\in \Z}\to \underbrace{\obar{1},\obar{2},\obar{1}}_{\in \Z/3\Z}\]
\end{example}
\begin{crl}
		If $I\subset R$ is prime, then $(I)\subset R[X]$ is prime.
\end{crl}

\begin{thm}[title = \texorpdfstring{$F[X]$}{F[X]} satisfies unique euclidean condition,label=12.8]
	If $a(X),b(X)\in F[X]$ where $F$ is a field. Then there exist \textbf{unique} $q(X),r(X)\in F[X]$ such that $\deg (r(X)) < \deg (b(X))$ (or $r(X)=0$) for which \[a(X)=q(X)\rdot b(X)+r(X)\]
\end{thm}
\Note The point of the above theorem being that elements in $F[X]$ have unique quotients and remainders, which doesn't always happen in a general Euclidean domain as $\Z$ is a Euclidean Domain with $N(n)=|n|$, e.g
\begin{align*}
7=3\rdot 2+1&\quad N(1)=1<N(2)\\
7=4\rdot 2-1&\quad N(-1)=1<N(2)
\end{align*}
\begin{proof}~\\
	Suppose $a(X)=q(X)\rdot b(X)+r(X)=q'(X)\rdot b(X)+r'(X)$, then
	\begin{align*}
	r(X)&=a(X)-q(X)\rdot b(X)\\
	r'(X)&=a(X)-q'(X)\rdot b(X)
	\end{align*}
	and $\deg(r),\deg(r')<\deg(b)$ (or they're both $0$ but then obviously they are unique).\\
	Consider
	\[r(X)-r'(X)=q'(x)\rdot b(X)-q(X)\rdot b(X) = [q'(X)-q(X)]\rdot b(X)\]
	Assume $q'-q\ne 0$ and $b\ne 0$, then since Euclidean domains are integral 	we have
	\begin{align*}
	\deg[(q'-q)\rdot b]=\deg(q'-q)+\deg(b)
	\end{align*}
	but also $(q'-q)\rdot b = r-r'$ for which we know
	\[\deg[r-r']<\deg b\]
	and hence a contradiction arises and it must be that $q'-q=0$ and so \[q'-q=0\implies q'=q\implies r=r'\qedhere\]
\end{proof}
The idea here behind this theorem and proof being that in regular Euclidean domains adding or subtracting alters the value of a norm while in polynomial rings, which has norm as the degree of the polynomial, it doesn't. This can be seen by considering that in the integers, if you start with $8$ and subtract off $1$, the norm is now $7$, but in the polynomial ring if you start with a polynomial of degree $8$ and subtract off a polynomial of strictly less degree (possibly even the same degree) then the norm (degree) does not change.
\begin{crl}
	Suppose $F,K$ are fields with $F\subset K$ and $a(X),b(X)\in F[X]$. \\
	Then the quotient and remainder polynomials of $a$ by $b$ are independent of field.
\end{crl}
\begin{proof}
	There exist $q(X),r(X)\in F[X]$ and $Q(X), R(X)\in K[X]$ with $\deg r< \deg b$ and $\deg R< \deg b$, such that
	\[a(X)=q(X)\rdot b(X)+r(X)\quad a(X)=Q(X)\rdot b(X)+R(X)\]
	But by \hyperref[thm:12.8]{the previous theorem}, there is uniqueness since $q,r\in K[X]$ it must mean that
	\[q(X)=Q(X)\quad r(X)=R(X)\qedhere\]
\end{proof}
\begin{crl}
	For fields $F,K$ with $F\subset K$, $\divs{b(X)}{a(X)}$ in $K[X]$ \textit{iff} $\divs{b(X)}{a(X)}$ in $F[X]$
\end{crl}
\begin{example}
	\[\divs{(X-1)}{X^2-1} \text{ in } \R[X] \text{ and so also in }\C[X]\]
	However note the case where $\divs{b(X)}{a(X)}$ in $K[X]$ but not in $F[X]$ e.g.
	\[\divs{(X-i)}{X^2+1} \text{ in } \C[X] \text{ but not } \R[X]\]
	Since $X^2+1$ has no non-trivial factors in $\R[X]$.
\end{example}
\begin{dfn}[title = Multivariable Polynoimal Ring]
	Let $R$ be a commutative ring with $1\ne 0$. \\
	The \textbf{polynomial ring in the variables }$X_1,\dots,X_n$ \textbf{with coefficients in R} is defined inductively as
	\[R[X_1,X_2,\dots,X_n]\coloneqq R[X_1,X_2,\dots,X_{n-1}][X_n]\]
	Concretely, think of $R[X_1,\dots,X_n]$ as finite sums of \textbf{monomials}, i.e 
	\[aX_1^{d_1}X_2^{d_1}\dots X_n^{d_n},\quad d_i\in \Z, \, d_i\ge 0\]
	e.g
	\[1+2XY+Y^2,2X-7X^3y+2XY^4+1\in \Z[X,Y]\]
\end{dfn}
\begin{dfn}[title = {Multi-Degree}]
	The \textbf{degree} of a monomial
	\[aX_1^{d_1}X_2^{d_1}\dots X_n^{d_n}\]
	is $d=d_1+d_2+\dots+d_n$.\\
	The \textbf{multi-degree} is $(d_1,d_2,d_3,\dots,d_n)$.\\
	The \textbf{degree} of a polynomial is the highest degree of any monomial in it.
\end{dfn}
\begin{prop}
	Let $R$ be an integral domain and
	\[p(X_1,\dots,X_n),q(X_1,\dots,X_n)\in R[X_1,X_2,\dots,X_n]\setminus\{0\}\] then
	\begin{enumerate}[label=(\arabic*)]
		\item $R[X_1,X_2,\dots,X_n]$ is an integral domain.
		\item $R[X_1,X_2,\dots,X_n]^\times = R^\times$
		\item $\deg[p\rdot q]=\deg p+\deg q$
	\end{enumerate}
\end{prop}
\end{document}