\documentclass[../Main.tex]{subfiles}
\setcounter{chapter}{13}

\begin{document}
\phantomsection
\chapter{L14: Factorization Techniques}
The goal of this lecture is to factor (or check for factors) of polynomials

\begin{prop}
	Let $F$ be a field and $p(X)\in F[X]$ a polynomial.\\
	$p(X)$ has a factor of degree one in $F[X]$ \textit{iff} $p(X)$ has a root in $F$, i.e $\exists \alpha \in F,\, p(\alpha)=0$.
\end{prop}
\begin{proof}~\\
	$\Longrightarrow$\\
	If $p(X)$ has a factor of degree one in $F[X]$ i.e $p(X)=(\alpha X-\beta) \rdot q(X),\, \alpha,\beta \in F$ with $\alpha\ne 0$ Then
	\[p\left(\frac{\beta}{\alpha}\right)=\left(\alpha \rdot \left(\frac{\beta}{\alpha}\right) - \beta \right)\rdot q\left(\frac{\beta}{\alpha}\right) = 0 \rdot  q\left(\frac{\beta}{\alpha}\right)=0\]
	$\Longleftarrow$\\
	Conversely, if $p(X)$ has a root $\alpha \in F$, then we can write 
	\[p(X)=q(X)\rdot (X-a)+r(X)\]
	where $r(X)=0$ or $\deg r(X)< \deg (X-\alpha) =1$ (i.e $r(X)\equiv r$ is a constant). Then, by substituting $\alpha$ we see
	\[p(\alpha)=q(\alpha)\rdot (\alpha-\alpha)+r \implies 0 = 0 +r \implies r=0\]
	and therefore $p(X) = q(X)\rdot (X-\alpha)$ where $(X-\alpha)$ is degree one factor we are looking for.
\end{proof}
\begin{crl}
	If $p(X)\in F[X]$ has (not necessarily distinct) roots $\alpha_1,\alpha_2,\dots,\alpha_k$, then $p(X)$ has
	\[(X-\alpha_1)\rdot (X-\alpha_2)\rdot \dots\rdot (X-\alpha_2)\]
	as a factor
\end{crl}
\begin{dfn}[title=Multiplicity]
	If $p(X)\in F[X]$ is divisible by $(X-\alpha)^k$, then we say that the root $\alpha$ has \textbf{multiplicity} $k$.
\end{dfn}
\begin{crl}
	If $\deg(p(X))=n$, then it has at most $n$ roots in $F$ (even counting with multiplicity).
\end{crl}
\begin{crl}
	If $p(X)\in F[X]$ and $\deg p = 2$ or $3$, then $p(X)$ is reducible \textit{iff} $p$ has a root in $F$.
\end{crl}
\begin{prop}
	Let 
	\[p(X)=a_0+a_1X+a_2X^2+\dots+a_nX^n\in \Z[X]\]
	If $\frac{r}{s}\in \Q$ is in lowest terms (i.e $\gcd(r,s)=1$) and $p\left(\frac{r}{s}\right)=0$, then $r|a_0$ and $s|a_1$.\\
	In particular, if $a_n=1$ (i.e $p$ is monic) and $p(d)\ne 0$ for all $d \in \Z$ such that $d|a_0$, then $p(X)$ has no roots in $\Q$.
\end{prop}
\begin{example}
	Let $p(X) = X^7-7X^2-2X+1$. Then check if $X=\pm 1$ are roots of $p(X)$:
	\begin{align*}
	&p(1)=1^7-7\rdot 1^2-2\rdot 1+1=-7\ne 0\\
	&p(-1)=(-1)^7-7\rdot (-1)^2-2\rdot (-1)+1=-5\ne 0
	\end{align*}
	Since neither are equal to $0$, then if $p(X)$ has any real roots, they are irrational.
\end{example}
\begin{proof}
	Let $\alpha=\frac{r}{s}$ be a root of a polynomial $p(X) \in \Z[X]$. Then one writes
	\begin{align*}
		&p\left(\frac{r}{s}\right)=a_0+a_1\rdot \left(\frac{r}{s}\right)+a_2\rdot \left(\frac{r}{s}\right)^2+\dots + a_n \left(\frac{r}{s}\right)^n\\
		\implies& 0 = a_0\rdot s^n +a_1\rdot r\rdot s^{n-1}+a_2\rdot r^2\rdot s^{n-2}+\dots+a_n\rdot r^n
	\end{align*}
	First isolating $r$, we get
	\begin{align*}
	a_n\rdot r^n =& -a_0 \rdot s^n-a_1\rdot r\rdot s^{n-1}-\dots-a_{n-1}\rdot r^{n-1}\rdot s \\
	=& -s \rdot \left(a_0\rdot s^{n-1}-a_1\rdot r\rdot s^{n-2}-\dots-a_{n-1}\rdot r^{n-1}\right)
	\end{align*}
	Since $\gcd(r,s)=1$ then it can only be that $s|a_n$.\\
	Similarly, isolating $s$, we get
	\begin{align*}
	a_0\rdot s^n =& -a_1\rdot r\rdot s^{n-1}-a_2\rdot r^2\rdot s^{n-2}-\dots-a_n\rdot r^n\\
	=& -r\rdot \left(a_1\rdot s^{n-1}-a_2\rdot r\rdot s^{n-2}-\dots-a_n\rdot r^{n-1}\right)
	\end{align*}
	Since $\gcd(r,s)=1$ then it can only be that $r|a_0$.
\end{proof}
\begin{example}
	Consider $p(X) = X^3+9X^2-2X+1$ with possible roots $X=\pm 1$. We check
	\begin{align*}
	&p(1)=1^3+9\rdot 1^2-2\rdot 1+1=9\ne 0\\
	&p(-1)=(-1)^3+9\rdot (-1)^2-2\rdot (-1)+1 = 11 \ne 0 
	\end{align*}
	Hence, $p(X)$ has no roots in $\Q$ and is thus \textbf{irreducible} over $\Q$.
\end{example}
\begin{claim}
	The polynomials $X^2-p,X^3-p \in \Z[X]$ where $p\in \Z$ is prime are irreducible over $\Q[X]$.
\end{claim}
\begin{proof}
	The only candidates for soluutions are $X=\pm 1, \pm p$. We check for $q(X)=X^2-P$:
	\begin{align*}
	&q(\pm 1) = (\pm 1)^2-p = 1-p \ne 0\\
	&q(\pm p) = (\pm p)^2-p = p\rdot (p-1)\ne 0
	\end{align*}
	The proof for $X^3-p$ is similar (you should check it yourself).
\end{proof}
\begin{example}
	Consider $p(X)=X^2+1$. This is irreducible over $\R[X]$ as one can check
	\begin{align*}
	1^2+1=2 \ne 0\\
	(-1)^2+1=2\ne 0
	\end{align*}
	On the other hand, it \textbf{is} reducible over $\modZ{2}[X]$ 
	\[1^2+1 \equiv 0 \Mod{2}\]
	Finally $X^2+X+1$ is irreducible over $\modZ{2}[X]$ as
	\begin{align*}
	0^2+0+1=1\ne 0\\
	1^2+1+1=1\ne 0
	\end{align*}
\end{example}
\begin{prop}
	Let $R$ be an integral domain and $I\subsetneq R$ a proper ideal. Let $p(X) \in R[X]$ be a non-constant, monic polynomial.\\
	If $\obar{p(X)} \in (R/I)[X]$ is irreducible into polynomials of strictly lesser degree, then $p(X)$ is irreducible in $R[X]$.
\end{prop}
\begin{proof}
	Suppose $p(X)$, a non-constant monic polynomial, is reducible in $R[X]$, say
	\[p(X)=a(X)\rdot b(X),\quad \deg a, \deg b< \deg p\]
	Since $p$ is monic then can also choose $a,b$ to be non-constant, monic polynomials, hence
	\[\obar{p(X)} = \obar{a(X)}\rdot \obar{b(X)} \in (R/I)[X]\qedhere\]
\end{proof}
\begin{example}~
	\begin{itemize}
		\item $p(X)=X^2+X+1$ is irreducible in $\modZ{2}[X]$ then it is irreducible in $\Z[X]$
		\item $p(X)=X^2+1$ is irreducible in $\Z[X]$ but \textbf{is} reducible in $(\modZ{2})[X]$
	\end{itemize}
	The second example shows the proposition cannot be an "if and only if" statement.
\end{example}
\textbf{\textcolor{BrickRed}{\underline{Warning}}}: There exist polynomials, e.g $X^4+1$ that are irreducible in $\Z[X]$ but are reducible in every $(\modZ{p})[X]$ for $p\in \Z $ prime.
\begin{example}
	Let $p(X,Y)\in \Z[X,Y] = (\Z[X])[Y]$, then
	\[\Z[X,Y]/(y\rdot \Z[X,Y]) \cong \Z[X]\]
	Specifically, $\obar{X^2+XY+1} \in \Z[X,Y]/(y\rdot \Z[X,Y])$. Since $X^2+1$ is an element of the coset $\obar{X^2+XY+1}$ and it is irreducible, then $X^2+XY+1$ is irreducible in $\Z[X,Y]$.
\end{example}
\newpage
\begin{thm}[title= Eisenstein's Criterion,label=eisen]
	Let $R$ be an integral domain and $P\subset R$ a prime ideal. Furthermore,
	\[q(X)=X^n+c_{n-1}X^{n-1}+\dots+c_1X+c_0\in R[X]\]
	Suppose $c_0,c_1,\dots,c_{n-1}\in P$ and $c_0\notin P^2$, then $q(X)$ is irreducible in $R[X]$.
\end{thm}
\begin{claim}
	$p(X) =X^4+3x^3-27X^2+9X+6$ is irreduicble
\end{claim}
\begin{proof}
	$3,-27,9,6\in 3\Z$ however $6\notin 9\Z$.
\end{proof}
\begin{proof}[Proof of Eisenstein's Criterion]
	Suppose $q(X)=a(X)\rdot b(X)$ where $a,b\in R[X]^\times$. Since $q$ is monic, we may take $a,b$ to be monic
	\begin{align*}
	a(X)=&X^k+a_{k-1}X^{k-1}+\dots+a_1X+a_0\\
	b(X)=&X^l+b_{l-1}X^{l-1}+\dots+b_1X+b_0
	\end{align*}
	where $l,k>0$.\\
	If $c_0,c_1,\dots,c_{n-1}\in P$, then
	\begin{align*}
	\obar{q(X)}=&\obar{X^n+c_{n-1}X^{n-1}+\dots+c_0}=\obar{X^n} \in (R/P)[X] \\
	 =&\obar{a(X)}\rdot \obar{b(X)}
	\end{align*}
	i.e $\obar{a(X)}\rdot \obar{b(X)} = \obar{X^n}$. Then necessarily
	\[\obar{a_0}\rdot \obar{b_0}=\obar{0} \implies a_0 \in P \text{ or } b_0\in P\]
	W.l.o.g let $a_0\in P$, then $a(X)\rdot b(X)$ can be written
	\begin{align*}
	&(X^k+a_{k-1}X^{k-1}+\dots+a_1X+a_0)\rdot (X^l+b_{l-1}X^{l-1}+\dots+b_1X+b_0) \\
	=&X^{k+l}+(a_{k-1}+b_{l-1})X^{k+l-1}+\dots+(a_1\rdot b_0+a_0\rdot b_1)X+a_0\rdot b_0
	\end{align*}
	Therefore $a_0\rdot b_1,a_1\rdot b_0\in P$ implying $a_1\in P$ or $b_0\in P$.\\
	If $a_1\in P$ then
	\[(a_2\rdot b_0+\underbrace{a_1\rdot b_1}_{\in P} + \underbrace{a_0\rdot b_2}_{\in P}) \implies a_2\rdot b_0\in P \implies a_2\in P \text{ or } b_0\in P \implies a_0\rdot b_0=c_0\in P^2\]
\end{proof}
\begin{example}
	$X^n-p$ is irreducible if $p$ is prime because $-p\in p\Z$ but $-p \notin p^2\Z$.
\end{example}
\begin{crl}
	$\sqrt[n]{p}\notin \Q$ for all  $n\ge 2$.
\end{crl}
\begin{example}
	Let $p(X)=X^4+1$ and notice that $1\notin P$ for any prime ideal (otherwise its the whole ring and not a prime ideal), therefore we can't apply \hyperref[thm:eisen]{Eisenstein's Criterion} directly.\\
	Consider
	\begin{align*}
	q(X)=&p(X+1)=(X+1)^4+1\\
	=& (X^4+4X^3+6X^2+4X+1)+1\\
	=& X^4+4X^3+6X^2+4X+2
	\end{align*}
	See that $2,4,6\in 2\Z$ but $2\notin 4\Z$, therefore we can apply \hyperref[thm:eisen]{Eisenstein's Criterion} to $q(X)$.
	Suppose $X^4+1=a(X)\rdot b(X)$ then 
	\[q(X)=(X+1)^4+1=a(X+1)\rdot b(X+1)\]
	i.e if $X^4+1$ is reducible then so is $q(X)$.\\
	But by \hyperref[thm:eisen]{Eisenstein's Criterion} $q(X)$ is irreducible, therefore $X^4+1$ is too.
\end{example}
\end{document}
