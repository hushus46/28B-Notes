\documentclass[../Main.tex]{subfiles}
\setcounter{chapter}{5}

\begin{document}
\chapter{More on Ideals}
Let $R$ be a ring with $1\ne 0$.\\
Recall that if $A\subset R$, then
	\[(A)= \bigcap_{\mathclap{\substack{I \subset R \text{ ideals}\\A\subset I} } } I\]
\begin{dfn}[title= Ring Multiplication]
	For fixed sets $A,B \subset R$, we define \textbf{ring multiplication} as
	\[A\rdot B \coloneqq \{a_1b_1 + \dots + a_nb_n \mid a_1,\dots, a_n \in A,\, b_1, \dots, b_n \in B,\, n\in \N\}\]
\end{dfn}
\begin{prop}[title = Characterization of ideal generated by a set]
	If $A\subset R$ is any subset, then:
	\begin{enumerate}
		\item $R\rdot A$ is the left ideal generated by $A$
		\item $A\rdot R $ is the right ideal generated by $A$
		\item $R\rdot A\rdot R$ is the (two-sided) ideal generated by $A$
	\end{enumerate}
\textit{Note:} If
\begin{itemize}
	\item $A=\emptyset$, then we say $RA=AR=RAR=\{0\}$
	\item $R$ is commutative, then $RA=AR=RAR$.
\end{itemize}
\end{prop}
\begin{proof}
	We will only check for the left ideal, the others follow similarly.\\
	First the subring criterion for $RA\subset R$\\
(i) $0=0\rdot a\in RA \implies RA \ne \emptyset$\\
(ii) Let $x,y \in RA$, then there exist
\begin{align*}
&r_1,\dots r_n \in R, \, a_1, \dots, a_n \in A \\
&r_1',\dots r_m' \in R, \, a_1', \dots, a_m' \in A 
\end{align*}
such that
\begin{align*}
x&=r_1a_1+r_2a_2+\dots +r_na_n\\
y&=r_1'a_1'+r_2'a_2'+\dots+r_m'+a_m'
\end{align*}
then
\begin{align*}
x-y&=(r_1a_2+\dots+r_na_n)-(r_1'a_1'+\dots+r_m'a_m')\\&=r_1a_1+\dots+r_na_n+(-r_1')a_1'+\dots+(-r_m')a_m'\in RA
\end{align*}
and
\begin{align*}
xy&=(r_1a_2+\dots+r_na_n)\rdot (r_1'a_1'+\dots+r_m'a_m') \\
&= (r_1a_1 r_1')a_1'+\dots+(r_1a_1r_m')a_m'\\
&+\vdots\\
&+(r_na_nr_1')a_1'+\dots +(r_na_nr_m')a_m' \in RA
\end{align*}
Then $RA$ is a subring.\\
To see $RA$ is an ideal: Let $r\in R, x\in RA$ as above.
\[r\rdot x=r\rdot (r_1a_2+\dots+r_na_n)=(rr_1)a_1+\dots+(rr_n)a_n \in RA\]
Moreover
\[A\subset RA \quad (1\in R \implies \forall a\in A, \, 1\rdot a=a\in RA)\]
So $RA$ is an ideal containing $A$ i.e
\[(A)\subset RA\]
On the other hand, if $I$ is a left ideal such that $A\subset I$, then $a\in A, r\in R \implies r\rdot a\in I$ which implies for any finite list $r_1,\dots,r_n\in R,\, a_1,\dots,a_n\in A$ 
\[r_1a_1,\dots,r_na_n\in I \implies r_1a_1+\dots+r_na_n\in I \implies RA\subset I\]
and since $(A)$ is a left ideal, we have
\[RA = (A)\]
and specifically this is the smallest ideal needed to contain $A$.
\end{proof}
\begin{prop}[title =\texorpdfstring{$I\rdot J \subset I \cap J$}{TEXT}]
	If $I,J \subset R$ are ideals, then $I\rdot J$ is an ideal, $I\rdot J \subset I \cap J$.
	\center
	\begin{tikzcd}[column sep=0em,row sep=1em,arrows=dash]
		&  R \arrow[dl]\arrow[dr] \\
		I \arrow[dr]& & J  \arrow[dl]\\
		& I \cap J  \arrow[d]  &\\     
		&IJ&
	\end{tikzcd}
\end{prop}
\textit{Note:} $I\rdot I=I^2, \dots, \underbrace{I\rdot I\rdot \dots \rdot I}_{n-\text{times}}=I^n$
\begin{example}
	Consider $2\Z, 3\Z \subset Z$, then
	\[2\Z\rdot 3\Z = \left\{ \left. \sum_{k=1}^n 2a_k\rdot 3b_k\right| a_k,b_k\in Z\right\} = \left\{ \left.6\left(\sum_{k=1}^n a_k\rdot b_k\right)\right| a_k,b_k\in Z\right\}=6\Z\]
	and
	\[2\Z\cap 3\Z = \{\underbrace{2n=3m}_{2|m,3|n}\}=6\Z\]
In this case we have $2\Z\rdot 3\Z=2\Z\cap 3\Z$.
\end{example}
\newpage
\begin{example}
	Consider the ring $R=\Z[X]$ with
	\begin{align*}
	(X)&\coloneqq\{p(X)\rdot x\mid p(X)\in R\}\\
	(X^2)&\coloneqq \{q(X)\rdot x^2\mid q(X)\in R\}
	\end{align*}
	Then
	\begin{align*}
	(X)\rdot (X^2)&=\{(p_1(X)\rdot X)\rdot (q_1(X)\rdot X^2)+\dots+(p_n(X)\rdot X)\rdot (q_n(X)\rdot X^2)\}\\ &= \{(p_1\rdot q_1(X)+\dots+p_n\rdot q_n(X))\rdot X^3\}=(X^3)
	\end{align*}
	On the other hand, since multiples of $X^2$ are also multiples of $X$, we get
	\[(X)\cap(X^2) = (X^2)\]
	and so
	\[(X)\rdot (X^2)=(X^3)\subsetneq(X)\cap(X^2)=(X^2)\]
	Since a multiple of $X^3$ is a multiple of $X^2$ but there is no multiple of $X^3$ which is equal to $aX^2$ for nonzero $a \in R$.
\end{example}
\section*{Large Ideals in \texorpdfstring{$R$}{TEXT} and Arithmetic in \texorpdfstring{$R$}{TEXT}}
Assume $R$ is a commutative ring w/ $1\ne 0$. \\
If $a\in R$, then
\[(a)=\{ra\mid a\in R\} \quad \text{(the "multiples" of a)}\]
e.g. $2\Z=\{2n\mid n\in Z\}=(2)$\\
\textit{Note:} We sometimes write
\[(a)=R\rdot a=a\rdot R\]
We also say that if $b\in (a)$, that $a$ \textbf{divides} $b$, i.e $a\mid b$.
\begin{claim}
	$b\in (a)$ \textit{iff} $(b) \subset (a)$
\end{claim}
\begin{proof}
	Let $b\in (a)$ then there exists $r\in R$ such that $b=r\rdot a$. In particular,
	\[c\in (b), \exists s\in R, \, c=s\rdot b=s\rdot (r\rdot a)=(s\rdot r)\rdot a\in (a)\implies (b) \subset (a)\]
	On the other hand, if $(b)\subset (a)$, then $b\in (b) \subset (a)$.
\end{proof}
\begin{dfn}[title = Prime Ideal]
	Let $R$ be a commutative ring.\\
	An ideal $P\ne R$ is called a \textbf{prime ideal} if for all $a,b\in R$ such that $a\rdot b\in P$, then either $a\in P$ or $b\in P$.
\end{dfn}
\newpage
\begin{example}~
\begin{itemize}
	\item$2\Z$ is prime
	\item$6\Z$ is \textbf{not} prime e.g $2\rdot 3 = 6 \in 6\Z$ but $2,3\notin 6\Z$
	\item $\{0\} \subset \Z$ is prime. If $a\rdot b=0, a,b \in \Z$ then either $a=0$ or $b=0$ (integral domain).
	\item $(x)\subset \R[x]$ is prime
	\item $(x^2)$ is \textbf{not}, e.g. $x\rdot x=x^2\in (x^2)$ but $x\notin (x^2)$.
\end{itemize}
\end{example}
\begin{prop}[title = \texorpdfstring{$R$}{PDFstring} integral if \texorpdfstring{$\{0\}$}{TEXT} prime]
	$R$ is an integral domain \textit{iff} $\{0\}$ is prime
\end{prop}
\begin{thm}[title = Prime Ideal \texorpdfstring{$\Longleftrightarrow R/P$}{TEXT} integral domain]
	Assume $R$ is commutative.\\
	An ideal $P\subset R$ is prime \textit{iff} $R/P$ is an integral domain.
\end{thm}
\begin{proof}~\\
	$\Rightarrow$\\
	Suppose $P$ is prime and  $\obar{a},\obar{b}\in R/P$ such that $\obar{a}\rdot \obar{b}=\obar{0}$.\\
	We want $\obar{a}=\obar{0}$ or $\obar{b}=\obar{0}$.\\
	Pick representatives $a\in \obar{a}, b\in \obar{b}$. This implies $\obar{a\rdot  b} = \obar{0}$, i.e $a\rdot b \in P$.\\
	But $P$ is prime, so either $a\in P$ or $b \in P$, i.e $\obar{a}=\obar{0},\obar{b}=\obar{0}$.\\
	$\Leftarrow$\\
	If $R/P$ is integral and $a\rdot b\in P$, then
	\[\obar{a\rdot b}=\obar{0}\implies \underbrace{\obar{a}=\obar{0} \text{ or } \obar{b} = \obar{0}}_{R/P \text{ integral }} \implies a\in P \text{ or } b \in P\]
\end{proof}
\end{document}