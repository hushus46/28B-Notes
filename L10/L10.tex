\documentclass[../Main.tex]{subfiles}
\setcounter{chapter}{9}

\begin{document}
\chapter{L10: Euclidean Domains and PIDs}
\begin{dfn}[title = Norm]
	Let $R$ be an integral domain.\\
	Any function
	\[N\colon R\to \Z^+\cup \{0\}\]
	such that $N(0)=0$ is called a \textbf{norm}.
\end{dfn}
\begin{example}
	The zero norm
	\begin{align*}
		N\colon R&\to \Z^+\cup \{0\} \\
		r&\mapsto 0
	\end{align*}
\end{example}
\begin{example}
	The absolute value norm on the integers
	\begin{align*}
	N\colon \Z&\to \Z^+\cup \{0\} \\
	n&\mapsto |n|
	\end{align*}
\end{example}
\begin{dfn}[title = {Euclidean Domain, Quotient, Remainder}]
	An integral domain $R$ is a \textbf{Euclidean domain} if it admits a norm $N$ such that for all $a,b\in R$ and $b\ne 0$, there exists $q,r\in R$ such that
	\[a=qb+r\]
	where $r=0$ or $N(b)>N(r)$ (i.e Euclidean domains have the \textit{familiar} division property known as the Euclidean condition).\\
	We call $q$ the \textbf{quotient} of $a$ by $b$ and $r$ the \textbf{remainder} of $a$ with respect to $b$.
\end{dfn}
What is nice about Euclidean domains is that you have the Euclidean Division Algorithm
\begin{align*}
a&=q_0b+r_0\\
b&=q_1r_0+r_1\\
r_0&=q_2r_1+r_2\\
&\vdots\\
r_{n-1}&=q_{n+1}r_n
\end{align*}
which must terminate because by the well ordering on the non-negative integers, you are constantly reducing the size of the remainder, so you must eventually reach $0$.
\[N(b)>N(r_0)>N(r_1)\dots >N(r_n)> N(r_{n+1})=N(0)=0 \]
\newpage
\begin{example}
	Fields $F$ are Euclidean domains with any norm $N$.\\
	If $a,b\in F, \, b\ne0$, then
	\[a=\underbrace{(a\rdot b^{-1})}_{\text{quotient}}\rdot\, b+ 0 \]
	which means in a field, you can always divide evenly.
\end{example}
\begin{example}
	The integers $\Z$ are a Euclidean domain with $N(a)=|a|$.
\end{example}
\begin{example}
	If $F$ is a field, the polynomial ring $F[X]$ is a Euclidean domain with norm $N(p)\coloneqq \deg(p)$. It's important to note that non-zero elements can have zero norm, as in this case, the constant polynomials have degree $0$.
\end{example}
\begin{proof}~\\
	Let $a(X),b(X)\in F[X]$ and $b(X)\ne 0$.\\
	We proceed by induction on $\deg(a)=N(a)$.\\
	If $a(X)=0$, then $0=0\rdot b(X)+0$.\\
	So we may assume $a(X)\ne 0$. If $\deg(a)<\deg(b)$, then 
	\[N(a)<N(b)\implies a(X)=0\rdot b(X)+a(X)\]
	which verifies the Euclidean condition.\\
	Now assume $\deg(a)\ge \deg(b)$, i.e
	\begin{align*}
	a(X)&= a_mX^m+a_{n-1}X^{m-1}+\dots+a_0\\
	b(X)&= b_nX^n+b_{n-1}X^{n-1}+\dots+b_0
	\end{align*}
	and since $b(X)\ne 0$ then $b_n\ne 0$ and since the coefficient ring is a field, we know $b_n^{-1} \in F$.\\
	Let 
	\[a'(X)=a(X)-\frac{a_m}{b_n}X^{m-n}\rdot b(X)\]
	then $\deg(a')<\deg(a)$ because we got rid of the term $a_mX^m$\\
	By induction on $\deg(a)$ there exist $q'(X), r'(X)$ such that $N(r')<N(b)$ or $r'(X)=0$ and
	\[a'=q'\rdot b+r'\]
	Hence we can write
	\begin{align*}
	a=&a'+\frac{a_m}{b_n}X^{m-n}\rdot b(X) \\
	a(X) =& \left[q'(X)\rdot b(X)+r'(X)\right]+\left[\frac{a_m}{b_n}X^{m-n}b(X)\right]\\
	=&\left[q'(X)+\frac{a_m}{b_n}X^{m-n}\right]b(X)+r'(X)
	\end{align*}
	and this also satisfies the Euclidean condition.
\end{proof}
\newpage
\begin{prop}[title = Euclidean domains are principal]
	Every ideal in a Euclidean domain is principal.
\end{prop}
\begin{proof}~\\
	If $I\subset R$ is a non-zero ideal, consider
	\[\mathcal{N} = \{N(a)\mid a\in I\} \subset \Z^+\cup \{0\}\]
	By the well-ordering principle, there exists $d\in I$ such that $N(d)=\min \mathcal{N}$. Clearly
	\[d\in I \implies (d)\subset I\]
	Conversely, suppose $a\in I$, then
	\[a=q\rdot d+r\]
	where $r=0$ or $N(r)<N(d)$.\\
	If $r=0$, then  
	\[a=q\rdot d \implies a\in (d)\implies I=(d)\]
	If $r\ne 0$, then $a-qd=r$. However
	\[a,d\in I\implies a-qd\in I \implies r\in I\]
	and because by construction $N(r)<N(d)$ this is impossible as $d$ is the element with minimum norm. Hence, $r=0$ and we go back to the previous situation.\\
	Therefore, $(d)=I$.
\end{proof}
\begin{crl}[title= Ideals in \texorpdfstring{$\Z$}{TEXT} are principal]
	Every ideal in $\Z$ is principal.
\end{crl}
Think about it like this: in the integers, if you consider the ideal generated by $2$ and $3$ and you know $3=2\rdot 1+1$, that means if $3$ is in the ideal with $2$, $1$ must also be in the ideal. So the $(2,3)=(1)$, so you have the whole ring. With similar logic, you can see that $(4,6)=(2)$. This extends to the general Euclidean domain as seen in Prop 10.1, as the ideal $(d)$ is the greatest common divisor.
\begin{dfn}[title = {Multiple, Divisor, GCD}]
	Let $R$ be a commutative ring with $1\ne 0$ and $a,b\in R$ such that $b\ne 0$.
	\begin{enumerate}[label=(\arabic*)]
		\item We say $a\in R$ is a \textbf{multiple} of $b$ if there exists an $r\in R$ such that
		\[a=r\rdot b\]
		We call $b$ a \textbf{divisor} of $a$, in this case, (i.e $b\mid a$).
		\item A \textbf{greatest common divisor} of $a,b\in R$ is $d\ne 0$ such that
		\begin{enumerate}[label=(\roman*)]
			\item $d\mid a, \, d\mid b$
			\item If $d'\mid a,\,  d'\mid b,$ then $d'\mid d$.
		\end{enumerate}
		We write $d=\gcd(a,b)$ or sometimes just $d=(a,b)$.
	\end{enumerate}
\end{dfn}
\textit{Recall} that $b\mid a$ if and only if $(a)\subset (b)$.
\begin{dfn}[title = Ideal GCD]
	Let $I = (a,b)\subset R$, then $d\in R$ is a \textbf{greatest common divisor} $d=\gcd(a,b)$ if
	\begin{enumerate}
		\item $I\subset (d)$
		\item If $I\subset (d')$, then $(d)\subset (d')$.
	\end{enumerate}
	In other words, $d\in R$ is a greatest common divisor of $a,b\in R$ if $(d)$ is the smallest principal ideal containing $(a,b)$.
\end{dfn}
\begin{prop}
	If $a,b\in R$ are nonzero, and $(a,b)=(d)$ then $d=\gcd(a,b)$
\end{prop}	
\begin{thm}[title = GCDs exist in Euclidean domains]
	If $R$ is a Euclidean domain, then greatest common divisors \textbf{always} exist
\end{thm}
\begin{proof}\[
	\begin{rcases*}
	a=q_0b+r_0\\
	b=q_1r_0+r_1\\
	r_0=q_2r_1+r_2\\
	\vdots\\
	r_{n-1}=q_{n+1}r_n
	\end{rcases*} \implies r_n = \gcd(a,b)\]
\end{proof}
\begin{dfn}[title = Principal Ideal Domain]
	A \textbf{principal ideal domain} (PID) is an integral domain in which every ideal is principal
\end{dfn}
\begin{thm}[title = Euclidean domain is PID is Integral domain]
	Every Euclidean domain is a PID, i.e
	\[\text{Integral domain} \supsetneq \text{ PID } \supsetneq \text{ Euclidean domain}\]
\end{thm}
\begin{thm}
	Let $R$ be a PID and $a,b\in R$ nonzero.
	If $(a,b)=(d)$ (this always exists in a PID), then
	\begin{enumerate}[label=(\arabic*)]
		\item $d$ is a greatest common divisor of $a$ and $b$.
		\item There exist $x,y\in R$ such that $d=ax+by$.
		\item $d$ is a unique to multiplication by a unit.
	\end{enumerate}
\end{thm}
\begin{claim}
	$\Z[X]$ is an integral domain BUT in particular $(2,X)$ is not principal therefore $\Z[X]$ is not a PID.
\end{claim}
\begin{proof}~\\
	Suppose it is principal, i.e $(2,X)=(p(X))$, then 
	\[2=q(X)p(X) \implies \deg p(X)=0\]
	i.e $p(X)\equiv a\in \Z$.\\
	Moreover $a\mid 2$ implies $a=\pm1,\pm2$. Also, $(2,X)\ne \Z[X]$ as for example
	\[3\ne \underbrace{2p(X)}_{3 \text{ is not even}}+\underbrace{X\rdot q(X)}_{\text{would need to be } 0}\]
	Then, $p(X)\ne \pm 1$ otherwise $(2,X)=(1)=\Z[X]$. Therefore $p(X)$ must be $\pm 2$.\\
	But $(2,X) \ne (2)$ because $X \ne 2\rdot q(X)$.Essentially, the issue is that $2$ has no multiplicative inverse in $\Z$ but the coefficient of $X$ is 1. So, nothing makes sense when $p(X)=\pm1,\pm2$ which means the initial assumption was false and $(2,X)$ is not principal.
\end{proof}
\begin{thm}[title = Nonzero primes ideals are maximal in PID]
	Every non-zero prime in a PID is maximal, e.g. in $\Z$, every prime is maximal.
\end{thm}
\begin{proof}
	Let $(p)\subset R$ be a nonzero prime in a PID.\\
	There exists a maximal ideal $M\subset R $ such that $(p)\subset M$.\\
	Since $R$ is a PID, then every ideal is principal, hence
	\[M=(m) \implies m\mid p \implies \exists r\in R, \, p=r\rdot m\]
	Because $(p)$ is prime either $r\in (p)$ or $m \in (p)$.\\
	If $m\in (p)$ then $(m)=(p)$. \\
	Suppose $r\in (p)$, say $r=s\rdot p,\, s\in R$. Then 
	\[p=r\rdot m=(s\rdot p)\rdot m \implies p\rdot (1-s\rdot m)=0\]
	Since $R$ is an integral domain and $p\ne 0$, then 
	\[1-sm=0\implies sm =1 \implies  m\in R^\times\]
	But then $(m) = R$, which means $(m)$ is not maximal, by definition. This is a contradiction and hence 
	\[(p)=(m)\]
	is maximal.
\end{proof}
\begin{thm}[title = If \texorpdfstring{$R[X]$}{R[X]} is PID then \texorpdfstring{$R$}{R} is field]
	If $R$ is a commutative ring such that $R[X]$ is a PID, then $R$ is a field.
\end{thm}
\begin{proof}~\\
	Suppose $R[X]$ is a PID (in particular, an integral domain), then $R\subset R[X]$ is an integral domain. We use a clever trick
	\[R[X]/(X) \cong R \implies (X) \text{ is prime} \implies (X) \text{ is maximal} \implies R \text{ is a field}\qedhere\]
\end{proof}
\end{document}